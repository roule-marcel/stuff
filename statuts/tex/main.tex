\documentclass[a4paper,12pt]{report}
\usepackage[utf8x]{inputenc}
\usepackage{palatino}
\usepackage{titlesec} %Nécessaire pour supprimer le mot chapitre
\usepackage[T1]{fontenc}
%\usepackage[francais]{babel}
\usepackage[french]{babel}
%\usepackage[english]{babel}
\usepackage[autolanguage,np]{numprint}
\usepackage{graphicx} %Pour insérer des images
\usepackage{float}
\usepackage{amsmath} 
\usepackage[absolute]{textpos}
\usepackage{textcomp}
\usepackage{multirow}
\usepackage[top=3.5cm,bottom=3.5cm,left=2.5cm,right=2.5cm]{geometry}
\usepackage{hyperref}
\usepackage{indentfirst}
\titleformat{\chapter}[hang]{\bf\huge}{\thechapter}{2pc}{} %Supprime le mot chapitre
\usepackage{eurosym}

\usepackage{color}

\renewcommand{\arraystretch}{1.5}
\setlength{\tabcolsep}{0.5cm}

\makeatletter\renewcommand{\thesection}{Article \@arabic\c@section \hspace{6mm}--}
\makeatother

\makeatletter\renewcommand{\thesubsection}{\@arabic\c@subsection}
\makeatother

\newcommand{\figpath}{figures}

\newcommand{\address}{Neuvitec 95, Mail Gay Lussac -- 95000 Neuville-sur-Oise, FRANCE}

%%Apport en nature:
%Attention, ne pas recourir à un commissaire aux apports n'est possible que si vous respectez les deux conditions cumulatives suivantes : - la valeur d'aucun apport en nature ne dépasse 30 000 euros ; - la valeur totale de tous les apports en nature ne dépasse pas la moitié du montant du capital social de la société. Par ailleurs, en cas de non désignation d'un commissaire aux apports, les associés restent, pendant cinq ans, solidairement responsables envers les tiers de la valeur qu'ils ont attribué aux apports.

\begin{document}

\begin{titlepage}
\thispagestyle{empty}
\newcommand{\HRule}[2]{\centering\rule{#1}{#2}}
\newlength{\logowidth}
\setlength{\logowidth}{2.5cm}
\newlength{\logohspace}
\setlength{\logohspace}{\linewidth}
\addtolength{\logohspace}{-3\logowidth}

\begin{center}
	
%	\large{2013-2014}

			
	%%%%%%%%%%%% Titre
	\begin{minipage}{0.8\linewidth}
	\vspace{3cm}
		\HRule{\linewidth}{0.5mm}\\
		\vspace{0.5em}
		\sc\Large
		Société Marcel SAS
		\HRule{\linewidth}{0.5mm}
	\end{minipage}
	%%%%%%%%%%%%
	\vspace{8cm}
%%	\vfill

	\large{\sc{Statuts}}
		
	\vspace{0.5cm}	
	\normalsize{SAS au capital de \textcolor{red}{XXX} Euros}
	
	\vspace{0.5cm}
	\normalsize{\address}
	
%	\vfill
% 	\includegraphics[height=5cm]{table.png} 
\end{center}

\end{titlepage}    


\begin{center}
	\begin{minipage}{0.8\linewidth}
		\center
		\rule{\linewidth}{0.5mm}\\
		\large{\sc{Société Marcel SAS}}\\
		\normalsize
		Société par actions simplifiée au capital de \textcolor{red}{XXX} Euros\\
		Siret : en cours d'immatriculation\\
		\normalsize{\address}\\
		\rule{\linewidth}{0.5mm}
	\end{minipage}
\end{center}
\vspace{5mm}
\noindent Les soussignés :\\\\
\textbf{Monsieur Fiack}, né le 16/02/1988 à HAGUENAU et domicilié 24 boulevard de l'Évasion, appartement 415, 95800 Cergy,\\\\
\textbf{Monsieur Fellus}, né le ../../.... à ... et domicilié ...,\\\\
ci-après désignés, les "associés", ont établi ainsi qu'il suit les statuts d'une société par actions simplifiée.

Les actionnaires de la présente société par actions simplifiée (ci-après, la "Société") sont au nombre de 2.

\section*{Préambule}
\textcolor{red}{Ceci est un préambule...}

Le présent préambule fait partie intégrante des statuts. En cas de différend sur l'interprétation des clauses statutaires, la volonté commune des parties, telle qu'elle y est indiquée, doit prévaloir à leur interprétation.

Ceci exposé, les soussignés ont établi les statuts de la société par actions simplifiée qu'ils sont convenus d'instituer entre eux.

\section{Forme}
Il existe entre les propriétaires des actions créées, une société par actions simplifiée régie par les lois et règlements en vigueur, et par les présents status.
Elle fonctionne sous la même forme avec un ou plusieurs associés.
Elle peut dans le cadre de la loi procéder à une offre au pulic de titres mais ne peut faire une offre au public de titre financier ni être admise sur un marché réglementé.

\section{Objet}
%La société a pour objet en France et à l'étranger :
%\begin{itemize}
%	\item La conception, la production, la vente et la location de plate-formes d'expérimentation et la prestation de services en mission de conseil, d'expertise, d'action et de formation pour les nouvelles technologies de l'information et de la communication,
%	\item et, plus généralement, toutes opérations commerciales, industrielles ou financières, mobilières ou immobilières,
%		pouvant se rattacher à l'objet social ou à tous objets connexes et susceptibles d'en faciliter le développement ou la réalisation ou à tous objets similaires, 
%		connexes ou complémentaires ou susceptibles d'en favoriser la réalisation, ou encore qui seraient de nature à faciliter, favoriser ou développer son commerce et son industrie.
%	\item La prise de participations par tous moyens dans toutes sociétés ou entreprises.
%\end{itemize}
La Société a, en France et à l'étranger, l'objet social suivant :

La conception, la production, la vente et la location de plate-formes d'expérimentation et la prestation de services en mission de conseil, d'expertise, d'action et de formation pour les nouvelles technologies de l'information et de la communication.

En outre, l'objet social comprend toutes opérations commerciales, industrielles ou financières, mobilières ou immobilières, qui s'y rapportent directement ou indirectement, susceptibles de lui être utiles ou d'en faciliter le développement ou la réalisation, ou à tous objets similaires, connexes ou complémentaires, ou encore qui seraient de nature à faciliter, favoriser ou développer son commerce et son industrie.

La Société peut agir directement, indirectement, seule ou en association, participation, groupement ou société, avec toutes autres personnes ou sociétés. Elle peut réaliser sous quelque forme que ce soit les opérations entrant dans son objet social.

\section{Dénomination}
La dénomination sociale est : \og \textbf{Société Marcel SAS} \fg{}\\
Dans tous les actes et documents émanant de la Société et destinés aux tiers, 
la dénomination sera précédée ou suivie immédiatement des mots écrits lisiblement \og Société par actions simplifée \fg{} 
ou des initiales \og S.A.S \fg{} et de l'énonciation du montant du capital social.

\section{Siège social}
Le siége social est fixé à l'adresse suivante : \address. 
%Il peut être transféré en tout endroit par décision du Président qui est habilité à modifier les statuts en conséquence.

Le Siège Social peut être transféré dans la zone géographique suivante : \textcolor{red}{XXX} par simple décision du Président. Lors d'un transfert décidé par le Président, celui-ci est autorisé à modifier les Statuts en conséquence, sous réserve de ratification par les associés en même temps que l'approbation des comptes de l'exercice.

En outre, le Siège Social peut être transféré en tout lieu en vertu d'une décision collective des associés, à la majorité prévue par les présents Statuts.

\section{Durée}
La durée de la Société est fixée à quatre-vingt-dix-neuf années à compter de la date de son immatriculatian au Registre du commerce et des sociétés.
%sauf les cas de dissolution anticipée ou de prorogation.

Un an au moins avant la date d'expiration de la Société, le Président provoquera une décision des associés afin de décider si la Société sera prorogée. 
À défaut d'une telle convocation des associés, conformément à l'article 1844-6 du Code civil, tout associé pourra demander au Président du tribunal de commerce, 
statuant sur simple requête, la désignation d'un mandataire pour que ce dernier obtienne une décision collective des associés sur l'éventuelle prorogation de la Société.

Cette durée peut, par décision de l'Assemblée Générale extraordinaire, être prorogée une ou plusieurs fois.

Les associés seront consultés pour décider de la prorogation selon les modalités prévues aux présents Statuts.

\section{Capital social}
Le capital social est fixé à la somme de \textcolor{red}{XXX €}, divisé en \textcolor{red}{XXX} actions ordinaires,
d'une valeur nominale de \textcolor{red}{XXX} (\textcolor{red}{XXX en chiffres}) euros chacune.
Les actions non libérées doivent l'être dans un délai de cinq ans à compter de l'immatriculation de la Société.

\section{Apports constitutifs du capital social}
L'ensemble des apports effectués à la Société s'élève à la somme de \textcolor{red}{XXX €} représentant :
\begin{enumerate}
	\item Les apports en numéraire pour un montant total de \textcolor{red}{XXX €}
	\item Les apports en nature évalués pour un montant total de \textcolor{red}{XXX €}
\end{enumerate}

\subsection{Apports en numéraire}
Les associés ont la possibilité de réaliser des apports en numéraire à la Société, qu'ils libèrent en tout ou partie sur un compte spécial.

La libération des apports des associés a fait l'objet d'une certification établie le \textcolor{red}{date} par l'établissement suivant : \textcolor{red}{BANQUE}

Les apports non libérés rendent la part correspondante des actions attribuées incessible tant que la libération n'est pas réalisée.

La libération du surplus interviendra en une ou plusieurs fois sur décision collective des associés.

Les actions non libérées doivent l'être dans un délai de cinq ans à compter de la date d'immatriculation de la Société.

Laurent Fiack, par ailleurs, fait apport à la Société d'une somme totale en numéraire de \textcolor{red}{XXX €}, libérée à \textcolor{red}{XXX \%}.

L'apport en numéraire de Laurent Fiack est rémunéré par l'attribution de \textcolor{red}{XXX} actions.

Jérôme Fellus, par ailleurs, fait apport à la Société d'une somme totale en numéraire de \textcolor{red}{XXX €}, libérée à \textcolor{red}{XXX \%}.

L'apport en numéraire de Jérôme Fellus est rémunéré par l'attribution de \textcolor{red}{XXX} actions.

\subsection{Apports en nature}
Les associés ont la possibilité de réaliser des apports en nature à la Société.

Laurent Fiack, par ailleurs, fait apport à la Société des biens suivants :

\textcolor{red}{PC trouvé dans la poubelle}

Lequel apport est estimé à la somme de \textcolor{red}{XXX €}

L'apport en nature de Laurent Fiack est rémunéré par l'attribution de \textcolor{red}{XXX} actions.

Jérôme Fellus, par ailleurs, fait apport à la Société des biens suivants :

\textcolor{red}{Paquet de gâteaux}

Lequel apport est estimé à la somme de \textcolor{red}{XXX €}.

L'apport en nature de Jérôme Fellus est rémunéré par l'attribution de \textcolor{red}{XXX} actions.

Les associés ont la possibilité de réaliser des apports en industrie à la Société.

Les actions en industrie sont nominatives, sans valeur nominale (ci-après, les "Actions en Industrie").

Les actions en Industrie ne participent pas à la formation du capital social.

Les apporteurs en industrie ont le droit de vote et aux partages des bénéfices et des pertes.

%À la constitution, les associés ont procédé aux apports suivants :
%\begin{itemize}
%	\item Monsieur \textbf{Fiack} :         XXX.OO€
%	\item Monsieur \textbf{Fellus} :         XXX.OO€
%\end{itemize}
%
%Soit une somme en numéraire de X.OOO € (X mille euros), correspondant à XXX actions de numéraire, d'une valeur nominale de XXX euros chacune, souscrites en totalité. 
%À la constitution, le capital est libéré à hauteur de XXX €.

\section{Modifications du capital social}
%\paragraph{I\hspace{4mm}--}Le capital social peut être augmenté par tous moyens et selon toutes modalités prévues par la loi.
%
%Le capital social est augmenté soit par émission d'actions ordinaires ou d'actions de préférence, soit par majoration du montant nominal des titres de capital existants. 
%Il peut également être augmenté par l'exercice de droits attachés à des valeurs mobilières donnant accès au capital, dans les conditions prévues par la loi.
%
%\paragraph{II\hspace{4mm}--}Le capital social peut être réduit, en vertu d'une décision de l'associé unique ou d'un vote des actionnaires, par la réduction du nombre d'actions, ou de leur valeur nominale, que la décision soit motivée ou non par des pertes.
%
%La réduction de capital se fera conforméménet aux dispositions légales en vigueur.
%
%Les actions en industrie seront réduites dans la même proportion que les actions en numéraire.
%
%%\paragraph{II\hspace{4mm}--}La réduction du capital est autorisée ou décidée par \textcolor{red}{l'actionnaire unique} délibérant dans les conditions prévues par la loi.
\subsection{Augmentation du capital social}

\paragraph{I\hspace{4mm}Augmentation du capital social par souscription d'actions nouvelles ou par augmentation de la valeur nominale d'actions existantes\\}

Le capital social peut être augmenté par les moyens de l'émission d'actions ordinaires ou d'actions de préférences, 
ainsi que par l'augmentation de la valeur nominale des titres de capital déjà existants.

L'augmentation de l'émission d'actions à souscrire en numéraire induit un droit de préférence à la souscription de ces actions 
au bénéfice des actionnaires en titre au moment de l'augmentation. Ce droit de préférence est réparti au prorata des participations 
des actionnaires dans le capital de la Société, aux conditions légales.

Cependant, les actionnaires ont le droit de renoncer à titre individuel à leur droit préférentiel de souscription. 
Par ailleurs, la décision d'augmentation du capital social peut avoir comme conséquence de supprimer le droit préférentiel de souscription 
dans les conditions légales, ainsi que par décision de la collectivité des associés ayant décidé de l'opération d'augmentation du capital. 
Cette décision peut porter sur une suppression totale ou partielle du droit de préférence, en faveur d'un ou de plusieurs associés dénommés, 
dans le respect des dispositions légales.

L'ensemble des associés a la possibilité de déléguer les pouvoirs nécessaires pour réaliser l'augmentation du capital en une ou plusieurs fois au Président, 
ainsi que les pouvoirs d'en déterminer les modalités, de constater leur réalisation et de procéder à la modification afférente des Statuts.

Toutefois, aucune souscription publique ne pourra être ouverte.

L'assemblée générale extraordinaire doit se prononcer sur le projet de résolution qui tend à réaliser une augmentation de capital 
lorsque celle-ci est ouverte aux salariés, dans les conditions prévues par la réglementation. 
Il n'est toutefois pas nécessaire que celle-ci se prononce dans les cas où la décision d'augmentation du capital résulte d'un apport en nature 
ou de l'émission préalable de valeurs mobilières donnant droit à l'attribution de titres représentant une quotité du capital.

\paragraph{II\hspace{4mm}Libération des actions lors d'une opération\\}

Lors de l'opération d'augmentation du capital par souscription d'actions, les actions en numéraire doivent être libérées au moins à 50 \% de leur valeur nominale.

Toutefois, les actions en numéraire doivent être libérées dans leur intégralité lorsque l'augmentation du capital est la conséquence d'une incorporation des réserves, 
d'une incorporation des bénéfices ou des primes d'émission, et pour partie d'un versement de sommes d'argent. 
De même, les actions émises en conséquence d'un apport en nature doivent être intégralement libérées.

Les associés doivent libérer le surplus en une ou plusieurs fois, sur décision du Président, 
dans un délai maximum de cinq ans à compter du jour où l'augmentation de capital est devenue définitive. 
Ils ont la faculté de procéder à des versements anticipés.

Les appels de fonds sont notifiés aux souscripteurs au moins 45 jours avant la date fixée pour chaque versement, 
par lettre recommandée avec demande d'avis de réception, adressée à chaque actionnaire.

La sanction du retard dans la libération des fonds dûs dans les dates décidées par le Président, 
est que la somme due devient de plein droit productive d'intérêts à 3\% annuel, à compter de la date d'exigibilité, 
sans préjudice des autres recours et sanctions prévus par la loi.

Conformément aux dispositions de l'article 1843-3 du Code civil, lorsqu'il n'a pas été procédé dans le délai légal aux appels de fonds 
pour réaliser la libération intégrale du capital, tout intéressé peut demander au Président du tribunal statuant en référé soit d'enjoindre sous astreinte 
aux dirigeants de procéder à ces appels de fonds, soit de désigner un mandataire chargé de procéder à ces formalités.

\subsection{Réduction du capital social}

Le capital social peut être réduit, en vertu d'une décision de l'associé unique ou d'un vote des actionnaires, par la réduction du nombre d'actions, 
ou de leur valeur nominale, que la décision soit motivée ou non par des pertes.

La réduction de capital se fera conformément aux dispositions légales en vigueur.

Les actions en industrie seront réduites dans la même proportion que les actions en numéraire.

\section{Libération des actions}
Lors d'une augmentation de capital, les actions de numéraire sont libérées, lors de la souscription, 
d'un quart au moins de leur valeur nominale et, le cas échéant, de la totalité de la prime d'émission.

La libération du surplus doit intervenir dans le délai de cinq ans à compter du jour où l'opération est devenue définitive en cas d'augmentation de capital. 
Les appels de fonds sont portés à la connaissance du ou des souscripteurs quinze jours au moins avant la date fixée pour chaque versement.


\section{Forme des actions}
Les actions sont obligatoirement nominatives. 
Elles donnent lieu à une inscription en compte individuel dans les conditions et selon les modalités prévues par la loi et les réglements en vigueur. 
Tout actionnaire peut demander à la Société la délivrance d'une attestation d'inscription en compte.

Les actions sont indivisibles à l'égard de la Société.

\section{Transmission des actions}
Les actions ne sont négociables qu'après l'immatriculation de la Société au Registre du Commerce et des Sociétés. 
En cas d'augmentation du capital, les actions sont négociables à compter de la réalisation de celle-ci. 
Les actions demeurent négociables après la dissolution de la Société et jusqu'à la clôture de la liquidation.

La propriété des actions résulte de leur inscription en compte individuel au nom du titulaire sur les registres tenus à cet effet au siège social.

La transmission des actions s'opère à l'égard de la Société et des tiers par un virement du compte du cédant au compte du cessionnaire, 
sur production d'un ordre de mouvement établi sur un formulaire fourni ou agréé par la Société et signé par le cédant ou son mandataire. 
L'ordre de mouvement est enregistré sur un registre coté et paraphé, tenu chronologiquement, dit \og registre des mouvements \fg{}.

La société est tenue de procéder à cette inscription et à ce virement dès réception de l'ordre de mouvement et, au plus tard, dans les huit jours qui suivent celle-ci. 
La société peut exiger que les signatures apposées sur l'ordre de mouvement soient certifiées par un officier ministériel.

Les bénéficiaires d'une mutation résultant d'une transmission d'actions doivent fournir à la Société tout document justifiant de leurs droits.

\section{Indivisibilité des actions}
La Société ne reconnaît qu'un seul propriétaire pour chacune des actions. À son égard, les actions sont indivisibles. 
Si certaines actions sont la propriété indivise de plusieurs personnes, 
alors les propriétaires indivis désignent un mandataire unique pour les représenter aux assemblées.

Toute action divisée en usufruit voit le droit de vote afférent dédié en toute matière au nu-propriétaire. 
Par exception, l'usufruitier prend les décisions concernant la répartition des bénéfices.

\section{Mise en location des actions}
Les actions sont susceptibles d'être louées dans les conditions prévues aux articles L. 239-1 à L. 239-4 du Code de commerce à une personne physique.

La location des actions est soumise à l'agrément du locataire par les associés, qui sera donné par une décision collective. 
En cas de refus d'agrément par la collectivité des associés, la location ne sera pas effective.

Les actions louées seront réputées délivrées au jour de l'inscription de la mention de la location et du nom du locataire 
à côté de celui du bailleur dans le registre des titres de la Société.

L'évaluation des actions louées s'opère sur la base de critères tirés des comptes sociaux, en début et fin de contrat.

Le bailleur des actions obéit aux mêmes règles qu'un nu-propriétaire et le locataire qu'un usufruitier. 
Ainsi, le droit de vote de l'action louée appartient au bailleur concernant les décisions collectives qui statuent sur les modifications statutaires. 
Pour les autres décisions, le droit de vote est exercé par le locataire.

\section{Droits et obligations attachés aux actions}
Toute action donne droit, dans les bénéfices et l'actif social, à une part nette proportionnelle à la quotité de capital qu'elle représente.

Chaque action donne en outre le droit au vote et à la représentation dans les assemblées générales, 
ainsi que le droit d'être informé sur la marche de la Société et d'obtenir communication de certains documents sociaux aux époques 
et dans les conditions prévues par la loi et les statuts.

La propriété d'une action comporte de plein droit adhésion aux statuts de la Société.

Les créanciers, ayants droit ou autres représentants d'un actionnaire ne peuvent, sous quelque prétexte que ce soit, 
requérir l'apposition de scellés sur les biens et valeurs sociales, ni en demander le partage ou la licitation.

%Chaque action donne droit à une fraction de l'actif social proportionnellement au nombre d'actions existantes. Chaque action donne droit à une voix au sein de tout vote et toute délibération.
%
%Chaque action de même catégorie donne droit à une fraction des bénéfices et de l'actif social, proportionnelle à la quotité du capital qu'elle représente.
%
%Les associés supportent les pertes à concurrence de leurs apports.
%
%Les droits et obligations sont attachés au titre, et se transmettent au cessionnaire en cas de circulation de l'action. Par ailleurs, la propriété d'une action emporte de plein droit pour l'associé propriétaire l'adhésion aux Statuts et aux résolutions régulièrement prises par les actionnaires.
%
%Chaque fois que pour exercer un droit quelconque, les propriétaires des actions en nombre inférieur à celui requis, pour exercer leur droit, font leur affaire personnelle du groupement ou, éventuellement, de l'achat ou vente des actions nécessaires.
%
%Le nu-propriétaire dispose du droit de vote, sauf pour les décisions qui sont relatives à l'affectation des bénéfices où ce droit de vote est réservé à l'usufruitier. Pour autant, le nu-propriétaire a le droit de participer à toute décision collective.
%
%Les associés ont le droit d'obtenir la délivrance d'une copie certifiée conforme des Statuts en vigueur le jour de leur demande, comportant en annexe, et le cas échéant, la liste des commissaires aux comptes en exercice. La délivrance a lieu au siège social et à la charge de l'associé demandeur.
%
%Les associés peuvent demander la communication des livres et des documents sociaux deux fois par an. Par ailleurs, deux fois par exercice, des associés représentant un vingtième du capital social peuvent poser des questions par écrit sur tout fait de nature à compromettre la continuité de l'exploitation au Président, dont la réponse doit être notifiée au(x) commissaire(s) aux comptes.

\section{Exclusion d'un associé}
Cette clause peut être modifiée à l'unanimité des associés.

Un associé personne morale qui fait l'objet d'une procédure de dissolution, de redressement ou de liquidation judiciaire est exclu de plein droit.

Un associé peut être par ailleurs exclu pour violation des Statuts, pour avoir :
\begin{itemize}
	\item commis des actes de nature à porter atteinte aux intérêts ou à l'image de marque de la Société,
	\item exercé une activité concurrente à celle de la Société,
	\item été révoqué de ses fonctions de mandataire social,
	\item créé une obstruction à des opérations sociales importantes.
\end{itemize}

L'exclusion est décidée par décision collective des associés.

L'exclusion d'un associé est décidée par un vote à la majorité absolue des membres présents ou représentés. Lors de ce vote, les associés voient leurs droits de vote réduits à une seule voix, quelle que soit leur participation en capital.

La décision d'exclure un associé nécessite que celui-ci en soit informé préalablement et régulièrement convoqué par le Président. L'information prend la forme d'une notification par lettre recommandée avec accusé de réception devant être envoyée deux semaines avant la date de prise de décision. La notification contient les éléments qui justifient l'exclusion, l'explication des faits accompagnée de pièces justificatives. Cette notification est communiquée, à l'identique, à tous les associés pour information. L'associé dont l'exclusion est envisagée a le droit de présenter ses arguments de défense aux autres associés ainsi qu'au Président et de se faire assister lors de la prise de décision à son encontre et peut également recourir, à ses frais, à un huissier de justice.

Ses arguments doivent figurer dans la décision finale des associés.

La décision d'exclusion statue sur le rachat des actions de l'associé exclu, elle permet de désigner ses acquéreurs, et les procédures statutaires habituelles en cas de cession telles que le droit de préemption ou d'agrément ne s'appliquent pas dans le cas de l'exclusion.

L'exclusion prononcée, l'associé exclu perd immédiatement son droit de participer et de voter aux réunions ou consultations d'associés, mais pas celui de percevoir les dividendes, et cède la totalité de ses actions dans un délai de trente jours à compter de l'exclusion aux autres associés au prorata de leur participation au capital. Le prix est fixé à l'amiable entre les parties. À défaut d'accord, le prix est déterminé dans les conditions prévues à l'article 1843-3 du Code civil.

Le registre des mouvements de titre de la Société est tenu à jour des cessions suivant une exclusion.

Si la cession des actions de l'associé exclu ou le versement du prix à celui-ci n'ont pas lieu dans les trente jours, la décision d'exclusion est nulle et de nul effet.

%\section{Gestion de la société}
%La société est représentée, dirigée et administrée par un Président.
%
%Le Président de la société est élu à la majorité simple par l'assemblée générale. 
%Le mandat du Président est renouvelable sans limitation.
%
%Le Président est nommé pour une durée de six ans.
%
%Les fonctions de Président prennent fin soit par le décès, la démission, la révocation, l'expiration de son mandat, 
%soit par l'ouverture à l'encontre de celui-ci d'une procédure de redressement ou de liquidation judiciaires.
%
%Le Président peut recevoir une rémunération dont les modalités sont fixées par la décision de nomination.

\section{Présidence}
La Société est gérée, administrée et représentée à l'égard des tiers par son Président, personne physique ou morale.

Les pouvoirs de Président seront exercés par Laurent Fiack, né(e) le 16/02/1988, et domicilié au \textcolor{red}{XXX}.

Le Président est nommé par les associés dans les conditions de vote des décisions collectives ordinaires.

Lorsqu'une personne morale a la qualité de Président, les dirigeants de celle-ci sont soumis aux mêmes conditions et obligations 
et encourent les mêmes responsabilités civiles et pénales que si ces personnes étaient Président en leur nom propre, 
sans préjudice de la responsabilité solidaire de la personne morale qu'ils dirigent.

Le Président est investi, en vertu de la loi, des pouvoirs les plus étendus pour agir en toutes circonstances au nom de la Société. 
Il les exerce dans la limite de l'Objet Social et dans la réserve des pouvoirs expressément attribués par la loi ou les Statuts aux associés.

La Société est engagée par tout acte du Président, même ne relevant pas de l'Objet Social, 
à moins qu'elle ne prouve que les tiers avaient connaissance du dépassement de l'Objet Social par l'acte du Président, 
ou qu'ils ne pouvaient pas l'ignorer compte tenu des circonstances. 
La seule publication des Statuts ne constitue cependant pas une preuve. 
Toute limitation des pouvoirs du Président par les Statuts est inopposable aux tiers.

Le Président assume la direction générale de la Société, sous sa responsabilité. 
Aussi, le Président peut accomplir tout acte de direction, de disposition, de gestion et d'administration de la Société. 
Ses pouvoirs sont limités par l'Objet Social et les prérogatives de décision des associés.

Le Président arrête les comptes à la fin de chaque exercice social. 
Il vérifie que les prescriptions légales et réglementaires sont respectées en la matière, 
il dresse l'inventaire des éléments de l'actif et du passif, du bilan, du compte de résultat et de l'annexe. 
Il établit le rapport de gestion obligatoire.

Le Président peut désigner des mandataires spéciaux par voie de subdélégation ou de substitution de pouvoirs pour un ou plusieurs objets déterminés, 
ou catégories d'opérations déterminées, en dehors des pouvoirs spécifiquement réservés à d'autres organes sociaux.

Le Président est responsable des infractions aux dispositions légales, des violations des Statuts, des fraudes qu'il commettrait durant sa gestion, 
envers la Société et les tiers.

Le Président a droit, pour le rémunérer de l'exercice de ses fonctions, à une rémunération fixe, ainsi qu'au remboursement de ses frais, 
sur fourniture des pièces justificatives.

Les modalités de traitement de cette rémunération seront fixées par décision collective des associés statuant à la majorité des deux tiers 
lors de l'approbation annuelle des comptes.

Le Président est nommé pour un mandat de cinq ans. Son mandat est renouvelable sans limitation.

Les fonctions du Président prennent fin à l'expiration de la durée de son mandat, ainsi qu'à la survenance d'évènements tels que son décès, 
sa démission, son empêchement pendant une durée supérieure à cinq mois, sa révocation, par la survenance d'une incapacité physique, mentale ou pénale, 
ou enfin du fait de l'ouverture d'une procédure de redressement ou de liquidation judiciaire.

Le Président est révocable à tout moment par les associés qui statuent dans les conditions de vote prévues pour les décisions ordinaires.

La révocation ne peut être effectuée que pour justes motifs.

Le Président doit être informé de la décision de révocation envisagée, par lettre recommandée avec accusé de réception, 
et avoir la possibilité de présenter ses observations aux associés avant l'intervention effective de la révocation.

Le Président peut quitter ses fonctions à tout moment, sous réserve qu'il respecte un préavis de \textcolor{red}{trente} jours, 
et qu'il notifie son départ par lettre recommandée avec accusé de réception. 
La Société peut demander au Président qui démissionne sans respecter le préavis 
ou qui est de mauvaise foi des dommages-intérêts forfaitaires à hauteur de \textcolor{red}{XXX €}.

Le Président remplaçant est désigné selon les mêmes modalités que pour la nomination du Président permanent, 
pour la durée qui reste à courir jusqu'à la fin du mandat de son prédécesseur.

\section{Directeurs généraux}
Le Président peut nommer un ou plusieurs directeurs généraux, personnes physiques qui portent le titre de directeur général 
ou de directeur général délégué, et qui peuvent ne pas être des actionnaires de la Société. 
Ils sont investis, sauf dispositions statutaires contraires inopposables aux tiers, des mêmes pouvoirs que le Président.

Les directeurs généraux ont un rôle d'assistance vis-à-vis du Président dans l'exercice de ses missions.

Les associés agréent le directeur général nommé par le Président par une décision collective votée à la majorité absolue.

La durée des fonctions du directeur général est fixée dans la décision de nomination, et ne peut excéder celle du Président, sauf en cas de démission, 
d'empêchement ou de décès de celui-ci. 
Dans ces derniers cas, le directeur général conserve ses fonctions jusqu'à ce qu'un Président temporaire soit nommé. 
Le mandat du directeur général est renouvelable sans limitation.

La rémunération des fonctions de directeur général est fixée par la décision collective qui le nomme.

Le directeur général peut démissionner de son mandat sous réserve de respecter un préavis d'un mois. 
Par ailleurs, il est révocable à tout moment, par décision collective et vote à la majorité absolue, 
sans nécessité de justes motifs et sans droit indemnisable, sans préjudice des règles du droit du travail.

\section{Conventions entre la société et ses dirigeants ou associés}
En application des dispositions de l'article L. 227-10 du Code de commerce, le Commissaire aux Comptes, s'il en est nommé un, présente un rapport sur les conventions, 
intervenues directement ou par personne interposée entre la Société et son Président et \textcolor{red}{actionnaire} unique.

En application des dispositions de l'article L. 227-11 du Code de commerce, les conventions portant sur les opérations courantes et conclues à des conditions normales qui, 
en raison de leur objet ou de leurs implications financières sont significatives pour les parties, sont communiquées au Commissaire aux Comptes.

Les interdictions prévues à l'article L. 225-43 du Code de commerce s'appliquent dans les conditions déterminées par cet article, au Président.

\section{Prérogatives décisionnelles}
Le Président, de la même manière que les directeurs généraux, ne peut pas accomplir seul certains actes ou opérations 
qui relèvent obligatoirement de la compétence des associés, leur accord préalable est nécessaire.

Sont notamment concernés les actes portant sur :
\begin{itemize}
	\item l'augmentation, la réduction ou l'amortissement du capital ;
	\item la nomination des commissaires aux comptes ;
	\item l'approbation des comptes annuels et aux bénéfices ;
	\item les opérations de fusion, scission, dissolution et transformation de la Société ;
	\item l'approbation des conventions réglementées ;
	\item l'exclusion d'un actionnaire ;
	\item les modifications statutaires ;
	\item l'agrément d'un cessionnaire d'actions ;
	\item l'apport partiel d'actifs ;
	\item la vente de fonds de commerce de la Société ;
	\item l'affectation du résultat,
	\item tout acte de disposition relatif à un fonds de commerce (vente, achat, nantissement, location-gérance, apport...) ;
	\item la création de filiale ;
	\item la conclusion de crédit-bail ;
	\item la constitution de garanties sur les biens sociaux.
\end{itemize}

Par ailleurs, les associés doivent également être préalablement consultés pour accord pour les opérations ou actes suivants :
\begin{itemize}
	\item toute prise de participation dans une société tierce ;
	\item tout investissement ou emprunt ;
	\item la subvention ou l'abandon de créances.
\end{itemize}

À cet effet, le président notifiera par écrit à tous les associés son intention de réaliser une de ces opérations. La notification devra indiquer :
\begin{itemize}
	\item la nature, le prix et les modalités de l'opération envisagée ;
	\item les conséquences financières et commerciales de l'opération ;
	\item les raisons pour lesquelles l'opération est diligentée.
\end{itemize}

%\section{Décisions collectives}
%L'assemblée générale est seule compétente pour prendre les décisions suivantes :
%\begin{itemize}
%	\item approbation des comptes annuels et affectation des résultats,
%	\item approbation des conventions réglementées,
%	\item nomination des Commissaires aux Comptes,
%	\item augmentation, amortissement et réduction du capital social,
%	\item transformation de la Société,
%	\item modification des statuts, sauf transfert du siège social.
%\end{itemize}
%
%Toutes autres décisions relèvent de la compétence du Président.

\section{Participation aux décisions collectives et conditions de majorité}
\subsection{Le droit de vote}
Chaque action donne droit à une voix.

Tous les actionnaires peuvent voter aux décisions collectives, que ce soit personnellement, à distance ou par l'intermédiaire d'un mandataire, quel que soit son nombre d'actions possédées, sous réserve de la déchéance encourue pour défaut de libération des versements exigibles sur les actions possédées. L'associé qui souhaite participer aux décisions collectives doit, toutefois, avoir préalablement inscrit en compte ses actions à son nom avant la date de la décision collective.

\subsection{Les conditions de majorité}
Sauf dispositions particulières des Statuts, les décisions sont prises selon les règles de majorité décrites au paragraphe suivant.

Les décisions ordinaires, qui ne modifient pas les statuts, ainsi que les décisions qui portent sur une augmentation du capital exclusivement par incorporation des réserves, des bénéfices ou des primes d'émission sont prises à la majorité absolue des associés présents ou représentés.

Les décisions qui sont votées à la majorité des deux tiers des voix des actionnaires présents ou représentés sont celles qui impliquent :
\begin{itemize}
	\item l'approbation des comptes ;
	\item l'affectation du résultat ;
	\item la nomination ou la révocation du Président ;
	\item la nomination d'un commissaire aux comptes ;
	\item la dissolution ou la liquidation de la Société ;
	\item l'augmentation et la réduction du capital ;
	\item la fusion, scission et apport partiel d'actifs ;
	\item l'agrément des cessions d'action ;
	\item le changement de siège social ;
	\item l'exclusion d'un actionnaire.
\end{itemize}

Enfin, sont prises à l'unanimité les décisions qui concernent :
\begin{itemize}
	\item l'adoption ou la modification des clauses statutaires relatives à l'inaliénabilité des actions, l'agrément des cessions d'actions, l'exclusion et la suspension d'un actionnaire,
	\item les modalités de vote et les conditions de majorité,
	\item la modification des règles relatives à l'affectation du résultat,
	\item le changement de forme de la Société.
\end{itemize}

\subsection{La tenue de l'assemblée générale}
Les décisions de la collectivité d'associés pourront être prises en assemblée générale, ou au besoin par vidéoconférence ou conférence par téléphone, ou par correspondance. Ces décisions collectives peuvent s'exprimer dans un acte signé par tous les associés ou par consultation écrite.

L'assemblée est convoquée par le Président, 30 jours au moins avant la date de réunion, aux frais de la Société, par tout procédé de communication par écrit ou électronique. Elle comporte l'indication de l'ordre du jour, de l'heure et du lieu de la réunion.

Les documents mentionnés ci-après sont communiqués à chaque associé avant toute décision collective et leur sont adressés avant toute assemblée, en même temps que le formulaire de vote à distance en cas de consultation écrite ou de vote par voie électronique, le cas échéant. Sont envoyés tous les documents utiles à l'information des associés, et plus particulièrement les informations relatives à l'ordre du jour et le texte des résolutions, ainsi que le rapport du Président et, le cas échéant, le rapport du commissaires aux comptes.

Si l'objet de l'assemblée est l'approbation des comptes sociaux alors les associés doivent recevoir en même temps que leur lettre de convocation à l'assemblée ou que la mise à disposition du formulaire de vote à distance l'ensemble des comptes annuels ou consolidés, le rapport sur la gestion du groupe, le tableau des résultats de la Société au cours de chacun des cinq derniers exercices clos.

L'assemblée est présidée par le Président, celui-ci peut être désigné au cours de l'assemblée, en cas d'absence de ce dernier, l'assemblée désigne un associé pour présider temporairement celle-ci. Un secrétaire est également désigné par les associés. Une feuille de présence est dressée par le Président et certifiée exacte pour chaque assemblée, émargée par chaque actionnaire présent ou représenté.

Le vote est valide si le taux de participation à la réunion est supérieur à 50 \% des titres.

Tout associé peut voter lors d'une consultation écrite ou, lors d'une assemblée ordinaire, par correspondance. À cet effet, la Société met à disposition des associés un formulaire qui est remis à ceux qui en font la demande. Ils complètent celui-ci, en établissant pour chaque résolution le sens de leur vote. Les associés disposent d'un délai maximal de 3 jours à compter de la réception des projets de résolution pour répondre et envoyer leur vote, par lettre recommandée avec accusé de réception ou télécopie. Au delà du délai de 3 jours, l'associé qui n'a pas envoyé le formulaire est réputé s'être abstenu.

Les associés pourront choisir de voter à distance sous la forme d'un courrier électronique. La Société devra obtenir le consentement de chaque actionnaire qui recevra les documents et formulaires de manière dématérialisée.

Toute délibération de l'assemblée générale des actionnaires ou toute consultation écrite est constatée dans un procès-verbal, que le Président dresse et signe.

Tous les procès-verbaux sont incorporés dans un registre spécial, conservé au siège social, registre qui est coté et paraphé, ou sur des feuilles mobiles, numérotées en continue et paraphées et tenus à jour conformément aux dispositions légales en vigueur.

Une fois la feuille remplie, partiellement ou totalement, elle est jointe aux autres feuilles utilisées, et toute modification telle qu'une addition, suppression, substitution ou inversion des feuille est interdite.

Le Président certifie conforme des copies ou extraits des délibérations des actionnaires, ou par le liquidateur si la Société est en liquidation.

Il est nécessaire que les associés ayant participé à la réunion signent le procès-verbal, l'acte ou le relevé des décisions dans un délai d'un mois. Le Président établit le procès-verbal et le signe également. Ce procès-verbal mentionne le vote de chaque actionnaire.

À chaque action est attaché un droit de vote, proportionnellement au capital représenté par l'action.

Les copies ou extraits du registre des assemblées sont certifiés conformes par le Président et le secrétaire, ou, le cas échéant, par le liquidateur de la Société.

Lorsqu'une décision provient du consentement unanime des associés et qu'elle est formalisée par un acte, celui-ci rappelle les documents et les informations qui ont été communiqués aux associés avant la prise de décision. Cet acte est signé par tous les associés et est retranscrit sur le registre spécial ou les feuilles numérotées.

%\section{Forme et modalités des décisions collectives}
%Les décisions collectives sont prises en assemblée générale. 
%Elles peuvent également faire l'objet d'une consultation écrite et être prises par tous moyens de télécommunication électronique.
%
%Toutefois, devront être prises en assemblée générale les décisions relatives à l'approbation des comptes annuels et à l'affectation des résultats, 
%aux modifications du capital social, à des opérations de fusion, scission ou apport partiel d'actif.
%
%Les décisions collectives prises en assemblée sont constatées par des procès-verbaux signés par le Président et le secrétaire et établis sur un registre spécial, 
%ou sur des feuillets mobiles numérotés.

\section{Conventions reglementées}
\subsection{Domaine}
Toute convention conclue entre la Société et son Président, un des directeurs généraux de la société ou un associé détenant plus de 10\% des droits de vote, ainsi qu'avec une société actionnaire contrôlant la Société est une convention réglementée (ci-après, les "Conventions Réglementées"). Les Conventions Réglementées sont soumises au contrôle des associés. Toutefois, les conventions qui portent sur des opérations courantes, conclues à des conditions normales ne sont pas concernées.

\subsection{Ratification}
Les Conventions Réglementées doivent toutefois être communiquées au commissaire aux comptes, s'il en existe un, sauf si en raison de leur objet ou de leur implication financière, elles ne sont significatives pour aucune des parties.

Le Président notifie les Conventions Réglementées au commissaire aux comptes s'il en existe un, dans un délai de deux mois à compter de leur conclusion.

Un rapport spécial (ci-après, le "Rapport") est rédigé par le Président, ou le commissaire aux comptes s'il en existe un, qui est rendu aux associés. Lors de l'approbation des comptes annuels, les associés statuent sur ce rapport.

En cas de consultation à distance, le rapport est joint aux documents adressés habituellement aux associés. En effet, tout associé a droit à obtenir communication de ce rapport.

Lorsque le vote des associés sur le Rapport se traduit par un refus de ratification, alors la Convention Réglementée est valide et cela n'entraîne pas sa nullité. Toutefois, le refus de ratification a pour conséquence que tout résultat dommageable résultant de la Convention Réglementée pour la Société est à la charge du Président, du dirigeant ou de l'associé contractant. En cas de pluralité des contractants, leur responsabilité est solidaire.

\subsection{Conventions interdites}
Les Conventions Réglementées, à peine de nullité du contrat, ne peuvent avoir pour objet, au bénéfice de la partie contractante ou de toute personne interposée telle que le Président, le directeur général ou l'associé, de :
\begin{itemize}
	\item contracter des emprunts auprès de la Société ou un découvert en compte courant ;
	\item de faire cautionner ou avaliser par la Société des engagements de la partie contractante envers les tiers.
\end{itemize}

\section{Comité d'entreprise}
Le cas échéant, un comité d'entreprise devra être constitué en application des dispositions des articles L. 2322-1 et suivants du Code du travail.

Les délégués du comité d'entreprise exercent les droits définis par l'article L. 2323-66 du Code du travail auprès du président ou de toute personne à laquelle le président aurait délégué ses pouvoirs relatifs au comité d'entreprise.

Préalablement à toute décision collective, le président adresse au comité d'entreprise les mêmes documents qu'aux salariés.

Le comité d'entreprise pourra demander d'inscrire des projets de résolution à l'ordre du jour de la réunion dans un délai de 30 jours avant la date prévue de la réunion. Le comité d'entreprise envoie sa demande par lettre recommandée avec accusé de réception. Sa demande doit comprendre le texte des projets de résolution, assorti d'un bref exposé des motifs.

Le président accuse réception de celle-ci dans un délai de 5 jours après sa réception, par lettre recommandée avec accusé de réception, qu'il envoie au comité d'entreprise.

\section{Commissaires aux comptes}
Un ou plusieurs commissaires aux comptes devront être nommés par décision collective des associés si les conditions de l'article L. 227-9-1 du Code de commerce sont remplies.

Les commissaires aux comptes sont nommés par décision collective des associés, à la majorité ordinaire. Peuvent être désignés également des commissaires aux comptes suppléants chargés de remplacer le titulaire en cas de refus, d'empêchement, de démission ou de décès.

Les commissaires aux comptes sont nommés pour six exercices. Leur fonction expire à la fin de l'assemblée générale statuant sur les comptes du dernier exercice.

Tout associé pourra demander à la Société de charger le commissaire aux comptes ou tout autre expert qu'il aura désigné pour accomplir des missions de contrôle comptable, d'audit ou d'expertise, pour la Société ou ses filiales.

\section{Exercice social}
Chaque exercice social a une durée d'une année, qui commence le \textbf{premier janvier} et finit le \textbf{trente et un décembre}. 
Par exception, le premier exercice commencera le jour de l'immatriculation de la Société au Registre du commerce et des sociétés et se terminera le \textbf{31 décembre 2016}.

\section{Comptes annuels}
Les opérations sociales figurent dans une comptabilité régulièrement tenue à jour.

Le Président dresse, à chaque fin d'exercice social :
\begin{itemize}
	\item l'inventaire ;
	\item les comptes annuels, conformément aux exigences du Code de commerce ;
	\item un rapport de gestion écrit qui expose, pour la période de l'exercice écoulé :
		\begin{itemize}
			\item la situation actuelle de la Société
			\item l'évolution prévisible
			\item les événements importants depuis la clôture de l'exercice
			\item les activités de recherche et développement
		\end{itemize}
	\item un bilan auquel est annexé un état des cautionnements, avals et garanties et sûretés consentis par la Société.
\end{itemize}

La présentation des comptes annuels et les méthodes d'évaluation ne peuvent pas être modifiées d'un exercice à l'autre, sauf changement exceptionnel dans la situation de la Société. Dans le cas d'une telle modification, l'annexe du bilan comporte toutes les justifications appropriées et le rapport de gestion et le rapport des commissaires aux comptes, le cas échéant.

Les commissaires aux comptes ont accès, dans le mois qui précède la convocation de l'assemblée ou à la consultation écrite des actionnaires appelée à statuer sur les comptes annuels, aux comptes annuels et au rapport de gestion.

Le Président convoque une décision collective visant à approuver les comptes de l'exercice écoulé dans les six mois après la clôture de l'exercice.

Cette décision collective permet en même temps au associés de statuer sur les Conventions Réglementées, en les approuvant ou les rejetant.

\section{Affectation et répartition du résultat}
Le compte de résultat récapitule les produits et les charges de l'exercice. La date d'encaissement et de paiement n'est pas prise en compte.

Le compte de résultat fait apparaître, par différence après déduction des amortissements et des provisions, le bénéfice ou la perte de l'exercice.

Du bénéfice de l'exercice, duquel on déduit les éventuelles pertes antérieures, 5\% des sommes sont prélevées pour les apporter au fonds de réserve légale. Ce prélèvement cesse d'être obligatoire lorsque le fonds de réserve aura atteint 10\% du capital social, et reprendra son cours si le montant de la réserve légale descendait ensuite en dessous de ce taux de 10\%.

Est également prélevé sur le bénéfice de l'exercice toute autre somme à porter en réserve en application de la loi.

Le bénéfice distribuable résulte du bénéfice de l'exercice auquel on déduit les pertes antérieures, les sommes portées en réserve, et auquel on ajoute le rapport bénéficiaire.

À moins d'une opération de réduction de capital, les bénéfices distribuables ne peuvent pas être distribués aux associés lorsque les capitaux propres sont ou deviendraient à la suite de cette distribution, inférieur au montant du capital auquel on additionne les réserves qui ne sont pas distribuables du fait de la loi ou des Statuts.

S'il existe des réserves facultatives, c'est-à-dire supérieures à 10\% du capital social, alors les associés peuvent décider de prélever des sommes sur celles-ci pour les distribuer, à titre ordinaire ou exceptionnel. Dans un tel cas, la décision de distribution précise sur quels postes de réserve les prélèvements ont lieu, après prélèvement du dividende sur le bénéfice distribuable.

L'écart de réévaluation ne peut pas être distribué. En revanche, il peut s'incorporer totalement ou partiellement au capital.

Un compte spécial est créé sur lequel les pertes sont inscrites après l'approbation des comptes. Elles pourront alors être imputées sur les bénéfices des exercices suivants.

Le bénéfice distribuable est soumis à la décision de l'assemblée générale, qui décide, sur proposition du Président, si celui-ci est réparti entre actionnaires en tant que dividende, affecté en réserves ou en amortissement du capital, ou reporté à nouveau.

Les associés décident collectivement des modalités de paiement des dividendes : en numéraire ou en actions de la Société.

Le paiement a lieu dans un délai de 3 mois à compter de la clôture de l'exercice.

L'actionnaire, pour recevoir les dividendes, présente son attestation d'inscription en compte. Les dividendes perçus régulièrement ne font l'objet d'aucune retenue ou de restitution, et restent acquis individuellement et définitivement aux actionnaires.

\section{Transformation de la société}
Les associés peuvent décider collectivement de transformer la forme de la Société. Le commissaire aux comptes, le cas échéant, rend un rapport qui atteste que les capitaux propres sont d'un montant au moins aussi grand que celui du capital social, sauf si la transformation vise la forme de société en nom collectif, auquel cas l'unanimité est nécessaire.

La transformation en société en commandite simple ou en société par actions est décidée à la majorité des deux tiers des associés présents ou représentés, et chaque associé qui devient commandité doit donner son accord.

La transformation en SARL nécessite également un vote à la majorité des deux tiers des associés présents ou représentés.

Toute transformation entraînant l'augmentation des engagements des associés ou la modification des clauses des Statuts exigeant l'unanimité, nécessite un vote à l'unanimité des associés.

\section{Dissolution anticipée de la société}
La dissolution anticipée est prononcée sur décision des actionnaires à la majorité des deux tiers prévue par les présents statuts.

La décision collective désigne les liquidateurs.

Si des pertes constatées dans les documents comptables ont eu pour conséquence que les capitaux propres de la Société atteignent un montant inférieur à la moitié du capital social, les actionnaires décident s'il y a lieu de dissoudre par anticipation la Société dans un délai de quatre mois après l'approbation des comptes révélant les pertes.

En cas de non dissolution de la Société, celle-ci réduit son capital d'un montant au moins égal à celui des pertes non imputées sur les réserves avant la clôture du deuxième exercice à la suite de celui qui a révélé les pertes, dans le cas où les capitaux propres n'ont pas été reconstitués à une valeur au moins égale à la moitié du capital social.

Que les associés décident de dissoudre la Société ou non, la résolution qu'ils adoptent doit être publiée.

Dans le cas où la décision collective n'a pas respecté les modalités ci-avant énoncées, ou si aucune décision n'a été prise, ou si les dispositions du quatrième paragraphe du présent article ne sont pas appliquées, alors tout intéressé pourra demander la dissolution de la Société devant un tribunal de commerce. La dissolution n'est opposable aux tiers qu'après la publication au Registre du commerce et des sociétés.

Le boni de liquidation, s'il en existe un, est réparti entre les associés proportionnellement au nombre de leurs actions.

Au moment de la dissolution, la Société est en liquidation. Sa dénomination est suivie, à partir de ce moment, des termes "société en liquidation". Le liquidateur est nommé par la décision collective de dissolution. Le liquidateur n'est pas obligatoirement un associé.

La collectivité des associés conserve ses attributions.

En fin de liquidation, les actionnaires sont convoqués pour statuer sur les comptes définitifs, le quitus des liquidateurs, la décharge de leur mandat, et constatent la clôture de la liquidation.

\section{Personnalité morale -- Immatriculation}
La Société ne jouira de la personnalité morale qu'à compter de son immatriculation au Registre du commerce et des sociétés de Cergy.

\section{Contestations}
Tous différends susceptibles de surgir pendant la durée de la société, ou au cours des opérations de liquidation, soit entre les actionnaires et les représentants légaux de la société, soit entre les actionnaires eux-mêmes, concernant les affaires sociales, l'exécution ou l'interprétation des présents Statuts seront jugés conformément à la loi et soumis à la juridiction compétente.

\section{Actes accomplis pour la société en formation}
Est dressé un état des actes accomplis pour la Société en formation, avec l'indication pour chacun de ces actes de l'engagement en résultant pour la Société. Cet état des actes est présenté aux associés, et est annexé aux Statuts, dont la signature emportera reprise de ces engagements par la société lorsqu'elle aura été immatriculée au Registre du Commerce et des Sociétés.

Les soussignés donnent rétroactivement le mandat à Laurent Fiack, domicilié au \textcolor{red}{XXX}, pour prendre, au nom et pour le compte de la Société, tous les engagements nécessaires jusqu'à son immatriculation au Registre du Commerce et des Sociétés.

\section{Publicité}
Le Président a tous pouvoirs pour remplir les formalités de publicité nécessaires imposées par la loi, dans un journal d'annonces légales, et au porteur d'un original, d'une copie ou d'un extrait des présentes pour effectuer toute autre formalité, notamment l'enregistrement des Statuts.

\section{Frais}
Tous les frais, droits et honoraires du fait de la constitution de la Société seront portés au compte "frais de premier établissement".

%\section{Inventaire -- Comptes annuels}
%
%Il est tenu une comptabilité réguliére des opérations sociales, conformément à la loi et aux usages du commerce. 
%À la clôture de chaque exercice, le Président dresse l'inventaire des divers éléments de l'actif et du passif existant à cette date.
%
%Il dresse également le bilan décrivant les éléments actifs et passifs et faisant apparaître de façon distincte les capitaux propres, 
%le compte de résultat récapitulant les produits et les charges de l'exercice, ainsi que l'annexe complétant et commentant l'information donnée par le bilan et le compte de résultat.
%
%Dans les six mois de la clôture de l'exercice ou, en cas de prolongation, dans le délai fixé par décision de justice, l'actionnaire doit statuer sur les comptes annuels, 
%au vu du rapport de gestion et des rapports du Commissaire aux Comptes s'il y a lieu.
%
%\section{Capitaux propores inférieurs à la moitié du capital social}
%Si, du fait des pertes constatées dans les documents comptables, les capitaux propres de la Société deviennent inférieurs à la moitié du capital social, 
%le Président doit, dans les quatre mois qui suivent l'approbation des comptes ayant fait apparaître ces pertes, décider en assemblée générale s'il y a lieu à dissolution anticipée de la Société.
%Si la dissolution n'est pas prononcée, le capital doit être, sous réserve des dispositions légales relatives au capital minimum, et dans le délai fixé par la loi, 
%réduit d'un montant égal à celui des pertes qui n'ont pu être imputées sur les réserves si, dans ce délai, les capitaux propres ne sont pas redevenus au moins égaux à la moitié du capital social.
%
%\section{Dissolution -- Liquidation}
%La société est dissoute dans les cas prévus par la loi et, sauf prorogatian, à l'expiration du terme fixé par les statuts, ou à la suite d'une décision de l'actionnaire.
%
%Un ou plusieurs liquidateurs sont alors nommés par cette décision collective.
%
%Le liquidateur représente la Société. 
%Il est investi des pouvoirs les plus étendus pour réaliser l'actif. 
%Il est habilité à payer les créanciers et à répartir le solde disponible.
%
%\section{Contestations}
%Les parties attribuent compétence au Président du Tribunal de commerce du lieu du siège social, 
%tant pour l'application des dispositions qui précédent, que pour le réglement de toutes autres difficultés.
%
%\section{Nomination des dirigeants}
%Le premier Président de la Société nommé aux termes des présents statuts, pour une durée indéterminée, est \textbf{Monsieur Fiack}.
%
%\section{Reprise des engagements accomplis pour le compte de la société en formation}
%Conformément à la loi, la Société ne jouira de la personnalité morale qu'à compter du jour de son immatriculation au Registre du commerce et des sociétés.
%L'état des actes accomplis au nom de la Société en formation, avec l'indication pour chacun d'eux de l'engagement qui en résulte pour la Société, est annexé aux présents statuts.
%La signature des présents statuts emportera reprise de ces engagements par la Société, lorsque celle-ci aura été immatriculée au Registre du commerce et des sociétés.
%
%\section{Formalités de publicité -- Pouvoirs -- Frais}
%Tous pouvoirs sont donnés au porteur d'un original ou d'une copie certifiée conforme des présentes pour effectuer l'ensemble des formalités légales relatives à la constitution de la société.
%Il a été fait cinq (5) exemplaires originaux des présents statuts.\\

\vspace{1cm}

Fait à....................................................., le.........................., en.......... exemplaires dont un pour chaque actionnaire, un pour l'enregistrement et deux pour le greffe.

\vspace{2cm}

\begin{minipage}{0.45\linewidth}
	\center
	\textbf{Monsieur Fiack}\\
	Président -- Associé
\end{minipage}
\begin{minipage}{0.45\linewidth}
	\center
	\textbf{Monsieur Fellus}\\
	Associé
\end{minipage}

%% Autres exemples de statuts :
%% - https://www.wonder.legal/fr/creation-modele/statuts-sas

\end{document}
