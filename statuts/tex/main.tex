\documentclass[a4paper,12pt]{report}
\usepackage[utf8x]{inputenc}
\usepackage{palatino}
\usepackage{titlesec} %Nécessaire pour supprimer le mot chapitre
\usepackage[T1]{fontenc}
%\usepackage[francais]{babel}
\usepackage[french]{babel}
%\usepackage[english]{babel}
\usepackage[autolanguage,np]{numprint}
\usepackage{graphicx} %Pour insérer des images
\usepackage{float}
\usepackage{amsmath} 
\usepackage[absolute]{textpos}
\usepackage{textcomp}
\usepackage{multirow}
\usepackage[top=3.5cm,bottom=3.5cm,left=2.5cm,right=2.5cm]{geometry}
\usepackage{hyperref}
\usepackage{indentfirst}
\titleformat{\chapter}[hang]{\bf\huge}{\thechapter}{2pc}{} %Supprime le mot chapitre
\usepackage{eurosym}

\usepackage{color}

\renewcommand{\arraystretch}{1.5}
\setlength{\tabcolsep}{0.5cm}

\makeatletter\renewcommand{\thesection}{Article \@arabic\c@section \hspace{6mm}--}
\makeatother

\newcommand{\figpath}{figures}

\newcommand{\address}{24 boulevard de l'Évasion -- 95800 CERGY, FRANCE}

\begin{document}

\begin{titlepage}
\thispagestyle{empty}
\newcommand{\HRule}[2]{\centering\rule{#1}{#2}}
\newlength{\logowidth}
\setlength{\logowidth}{2.5cm}
\newlength{\logohspace}
\setlength{\logohspace}{\linewidth}
\addtolength{\logohspace}{-3\logowidth}

\begin{center}
	
%	\large{2013-2014}

			
	%%%%%%%%%%%% Titre
	\begin{minipage}{0.8\linewidth}
	\vspace{3cm}
		\HRule{\linewidth}{0.5mm}\\
		\vspace{0.5em}
		\sc\Large
		Société Marcel SAS
		\HRule{\linewidth}{0.5mm}
	\end{minipage}
	%%%%%%%%%%%%
	\vspace{8cm}
%%	\vfill

	\large{\sc{Statuts}}
		
	\vspace{0.5cm}	
	\normalsize{SAS au capital de \textcolor{red}{XXX} Euros}
	
	\vspace{0.5cm}
	\normalsize{\address}
	
%	\vfill
% 	\includegraphics[height=5cm]{table.png} 
\end{center}

\end{titlepage}    


\begin{center}
	\begin{minipage}{0.8\linewidth}
		\center
		\rule{\linewidth}{0.5mm}\\
		\large{\sc{Société Marcel SAS}}\\
		\normalsize
		Société par actions simplifiée au capital de \textcolor{red}{XXX} Euros\\
		Siret : en cours d'immatriculation\\
		\normalsize{\address}\\
		\rule{\linewidth}{0.5mm}
	\end{minipage}
\end{center}
\vspace{5mm}
\noindent Les soussignés :\\\\
\textbf{Monsieur Fiack}, né le 16/02/1988 à HAGUENAU et domicilié 24 boulevard de l'Évasion, appartement 415, 95800 Cergy,\\\\
\textbf{Monsieur Fellus}, né le ../../.... à ... et domicilié ...,\\\\
ont établi ainsi qu'il suit les statuts d'une société par actions simplifiée.

\section{Forme}
Il existe entre les propriétaires des actions créées, une société par actions simplifiée régie par les lois et règlements en vigueur, et par les présents status.
Elle fonctionne sous la même forme avec un ou plusieurs associés.
Elle peut dans le cadre de la loi procéder à une offre au pulic de titres mais ne peut faire une offre au public de titre financier ni être admise sur un marché réglementé.

\section{Objet}
La société a pour objet en France et à l'étranger :
\begin{itemize}
	\item \textcolor{red}{XXX}
	\item et, plus généralement, toutes opérations commerciales, industrielles ou financières, mobilières ou immobilières,
		pouvant se rattacher à l'objet social ou à tous objets connexes et susceptibles d'en faciliter le développement ou la réalisation ou à tous objets similaires, 
		connexes ou complémentaires ou susceptibles d'en favoriser la réalisation, ou encore qui seraient de nature à faciliter, favoriser ou développer son commerce et son industrie.
	\item La prise de participations par tous moyens dans toutes sociétés ou entreprises.
\end{itemize}

\section{Dénomination}
La dénomination sociale est : \og \textbf{Société Marcel SAS} \fg{}\\
Dans tous les actes et documents émanant de la Société et destinés aux tiers, 
la dénomination sera précédée ou suivie immédiatement des mots écrits lisiblement \og Société par actions simplifée \fg{} 
ou des initiales \og S.A.S \fg{} et de l'énonciation du montant du capital social.

\section{Siège social}
Le siége social est fixé \address. 
Il peut être transféré en tout endroit par décision du Président qui est habilité à modifier les statuts en conséquence.

\section{Durée}
La durée de la Société est fixée à quatre-vingt-dix-neuf années à compter de la date de son immatriculatian au Registre du commerce et des sociétés, 
sauf les cas de dissolution anticipée ou de prorogation.

\section{Apports}
À la constitution, les associés ont procédé aux apports suivants :
\begin{itemize}
	\item Monsieur \textbf{Fiack} :         XXX.OO€
	\item Monsieur \textbf{Fellus} :         XXX.OO€
\end{itemize}

Soit une somme en numéraire de X.OOO € (X mille euros), correspondant à XXX actions de numéraire, d'une valeur nominale de XXX euras chacune, souscrites en totalité. 
À la constitution, le capital est libéré à hauteur de XXX €.

\section{Capital social}
Le capital social est fixé à la somme de XXX mille euros, divisé en XXX actions ordinaires,
d'une valeur nominale de XXX (XXX en chiffres) euros chacune, entiérement souscrites et libérées.

\section{Modifications du capital social}
\paragraph{I\hspace{4mm}--}Le capital social peut être augmenté par tous moyens et selon toutes modalités prévues par la loi.

Le capital social est augmenté soit par émission d'actions ordinaires ou d'actions de préférence, soit par majoration du montant nominal des titres de capital existants. 
Il peut également être augmenté par l'exercice de droits attachés à des valeurs mobiliéres donnant accés au capital, dans les conditions prévues par la loi.

\paragraph{II\hspace{4mm}--}La réduction du capital est autorisée ou décidée par l'actionnaire unique délibérant dans les conditions prévues par la loi.

\section{Libération des actions}
Lors d'une augmentatian de capital, les actions de numéraire sont libérées, lors de la souscription, 
d'un quart au moins de leur valeur nominale et, le cas échéant, de la totalité de la prime d'émission.

La libération du surplus doit intervenir dans le délai de cinq ans à compter du jour ou l'opération est devenue défnitive en cas d'augmentation de capital. 
Les appels de fonds sont portés à la connaissance du ou des souscripteurs quinze jours au moins avant la date fixée pour chaque versement.


\section{Forme des actions}
Les actions sont obligatoirement nominatives. 
Elles donnent lieu à une inscription en compte individuel dans les conditions et selon les modalités prévues par la loi et les réglements en vigueur. 
Tout actionnaire peut demander à la Société la délivrance d'une attestation d'inscription en compte.

\section{Transmission des actions}
Les actions ne sont négociables qu'aprés l'immatriculation de la Société au Registre du Commerce et des Sociétés. 
En cas d'augmentation du capital, les actions sont négociables à compter de la réalisation de celle-ci. 
Les actions demeurent négociables aprés la dissolution de la Société et jusqu'à la clôture de la liquidation.

La propriété des actions résulte de leur inscription en compte individuel au nom du titulaire sur les registres tenus à cet effet au siège social.

La transmission des actions s'opère à l'égard de la Société et des tiers par un virement du compte du cédant au compte du cessionnaire, 
sur production d'un ordre de mouvement établi sur un formulaire fourni ou agréé par la Société et signé par le cédant ou son mandataire. 
L'ordre de mouvement est enregistré sur un registre coté et paraphé, tenu chronologiquement, dit \og registre des mouvements \fg{}.

La société est tenue de procéder à cette inscription et à ce virement dès réception de l'ordre de mouvement et, au plus tard, dans les huit jours qui suivent celle-ci. 
La société peut exiger que les signatures apposées sur l'ordre de mouvement soient certifiées par un officier ministériel.

\section{Droits et obligations attachés aux actions}
Toute action donne droit, dans les bénéfices et l'actif social, à une part nette proportionnelle à la quotité de capital qu'elle représente.

Chaque action donne en outre le droit au vote et à la représentation dans les assemblées générales, 
ainsi que le droit d'être informé sur la marche de la Société et d'obtenir communication de certains documents sociaux aux époques et dans les conditions prévues par la loi et les statuts.

La propriété d'une action camporte de plein droit adhésion aux statuts de la Société.

Les créanciers, ayants droit ou autres représentants d'un actionnaire ne peuvent, sous quelque prétexte que ce soit, 
requérir l'apposition de scellés sur les biens et valeurs sociales, ni en demander le partage ou la licitation.

\section{Gestion de la société}
La société est représentée, dirigée et administrée par un Président.

Le Président de la société est élu à la majorité simple par l'assemblée générale. 
Le mandat du Président est renouvelable sans limitation.

Le Président est nommé pour une durée de six ans.

Les fonctions de Président prennent fin soit par le décés, la démission, la révocation, l'expiration de son mandat, 
soit par l'ouverture à l'encontre de celui-ci d'une procédure de redressement ou de liquidation judiciaires.

Le Président peut recevoir une rémunération dont les modalités sont fixées par la décision de nomination.

\section{Conventions entre la société et ses dirigeants ou associés}
En application des dispositions de l'article L. 227-1O du Code de commerce, le Commissaire aux Comptes, s'il en est nommé un, présente un rapport sur les conventions, 
intervenues directement ou par persanne interposée entre la Société et son Président et actionnaire unique.

En application des dispositions de l'article L. 227-11 du Code de commerce, les conventions portant sur les opérations courantes et conclues à des conditions normales qui, 
en raison de leur objet ou de leurs implications financières sont significatives pour les parties, sont communiquées au Commissaire aux Comptes.

Les interdictions prévues à l'article L. 225-43 du Code de cammerce s'appliquent dans les conditions déterminées par cet article, au Président.


\section{Décisions collectives}
L'assemblée générale est seule compétente pour prendre les décisions suivantes :
\begin{itemize}
	\item approbation des comptes annuels et affectation des résultats,
	\item approbation des conventions réglementées,
	\item nomination des Commissaires aux Comptes,
	\item augmentation, amortissement et réduction du capital social,
	\item transformation de la Société,
	\item modification des statuts, sauf transfert du siège social.
\end{itemize}

Toutes autres décisions relèvent de la compétence du Président.

\section{Forme et modalités des décisions collectives}
Les décisions collectives sont prises en assemblée générale. 
Elles peuvent également faire l'objet d'une consultation écrite et être prises par tous moyens de télécommunication électronique.

Toutefois, devront être prises en assemblée générale les décisions relatives à l'approbation des comptes annuels et à l'affectation des résultats, 
aux modifications du capital social, à des opérations de fusion, scission ou apport partiel d'actif.

Les décisions collectives prises en assemblée sont constatées par des procés-verbaux signés par le Président et le secrétaire et établis sur un registre spécial, 
ou sur des feuillets mobiles numérotés.

\section{Exercice social}
Chaque exercice social a une durée d'une année, qui commence le \textbf{premier janvier} et finit le \textbf{trente et un décembre}. 
Par exception, le premier exercice commencera le jour de l'immatriculation de la Société au Registre du commerce et des sociétés et se terminera le \textbf{31 décembre 2016}.

\section{Inventaire -- Comptes annuels}

Il est tenu une comptabilité réguliére des opérations sociales, conformément à la loi et aux usages du commerce. 
À la clôture de chaque exercice, le Président dresse l'inventaire des divers éléments de l'actif et du passif existant à cette date.

Il dresse également le bilan décrivant les éléments actifs et passifs et faisant apparaître de façon distincte les capitaux propres, 
le compte de résultat récapitulant les produits et les charges de l'exercice, ainsi que l'annexe complétant et commentant l'information donnée par le bilan et le compte de résultat.

Dans les six mois de la clôture de l'exercice ou, en cas de prolongation, dans le délai fixé par décision de justice, l'actionnaire doit statuer sur les comptes annuels, 
au vu du rapport de gestion et des rapports du Commissaire aux Comptes s'il y a lieu.

\section{Capitaux propores inférieurs à la moitié du capital social}
Si, du fait des pertes constatées dans les documents comptables, les capitaux propres de la Société deviennent inférieurs à la moitié du capital social, 
le Président doit, dans les quatre mois qui suivent l'approbation des comptes ayant fait apparaître ces pertes, décider en assemblée générale s'il y a lieu à dissolution anticipée de la Société.
Si la dissolution n'est pas prononcée, le capital doit être, sous réserve des dispositions légales relatives au capital minimum, et dans le délai fixé par la loi, 
réduit d'un montant égal à celui des pertes qui n'ont pu être imputées sur les réserves si, dans ce délai, les capitaux propres ne sont pas redevenus au moins égaux à la moitié du capital social.

\section{Dissolution -- Liquidation}
La société est dissoute dans les cas prévus par la loi et, sauf prorogatian, à l'expiration du terme fixé par les statuts, ou à la suite d'une décision de l'actionnaire.

Un ou plusieurs liquidateurs sont alors nommés par cette décision collective.

Le liquidateur représente la Société. 
Il est investi des pouvoirs les plus étendus pour réaliser l'actif. 
Il est habilité à payer les créanciers et à répartir le solde disponible.

\section{Contestations}
Les parties attribuent compétence au Président du Tribunal de commerce du lieu du siège social, 
tant pour l'application des dispositions qui précédent, que pour le réglement de toutes autres difficultés.

\section{Nomination des dirigeants}
Le premier Président de la Société nommé aux termes des présents statuts, pour une durée indéterminée, est \textbf{Monsieur Fiack}.

\section{Reprise des engagements accomplis pour le compte de la société en formation}
Conformément à la loi, la Société ne jouira de la personnalité morale qu'à compter du jour de son immatriculation au Registre du commerce et des sociétés.
L'état des actes accomplis au nom de la Société en formation, avec l'indication pour chacun d'eux de l'engagement qui en résulte pour la Société, est annexé aux présents statuts.
La signature des présents statuts emportera reprise de ces engagements par la Société, lorsque celle-ci aura été immatriculée au Registre du commerce et des sociétés.

\section{Formalités de publicité -- Pouvoirs -- Frais}
Tous pouvoirs sont donnés au porteur d'un original ou d'une copie certifiée conforme des présentes pour effectuer l'ensemble des formalités légales relatives à la constitution de la société.
Il a été fait cinq (5) exemplaires originaux des présents statuts.\\

Fait à NEUVILLE,

Le ../../2016.\\

\vspace{2cm}

\begin{minipage}{0.45\linewidth}
	\center
	\textbf{Monsieur Fiack}\\
	Président -- Associé
\end{minipage}
\begin{minipage}{0.45\linewidth}
	\center
	\textbf{Monsieur Fellus}\\
	Associé
\end{minipage}
\end{document}
