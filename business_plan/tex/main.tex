\documentclass[a4paper,12pt]{report}
\usepackage[utf8x]{inputenc}
\usepackage{palatino}
\usepackage{titlesec} %Nécessaire pour supprimer le mot chapitre
\usepackage[T1]{fontenc}
%\usepackage[francais]{babel}
\usepackage[french]{babel}
%\usepackage[english]{babel}
\usepackage[autolanguage,np]{numprint}
\usepackage{graphicx} %Pour insérer des images
\usepackage{float}
\usepackage{amsmath} 
\usepackage[absolute]{textpos}
\usepackage{textcomp}
\usepackage{multirow}
\usepackage[top=3.5cm,bottom=3.5cm,left=2.5cm,right=2.5cm]{geometry}
\usepackage{hyperref}
\usepackage{indentfirst}
\titleformat{\chapter}[hang]{\bf\huge}{\thechapter}{2pc}{} %Supprime le mot chapitre
\usepackage{eurosym}
\usepackage{wrapfig}

\usepackage{color}

\renewcommand{\arraystretch}{1.5}
\setlength{\tabcolsep}{0.5cm}

\makeatletter\renewcommand{\thesection}{\@arabic\c@section}
\makeatother

\newcommand{\figpath}{figures}

\newcommand{\address}{Neuvitec 95, Mail Gay Lussac -- 95000 Neuville-sur-Oise, FRANCE}

\begin{document}

\begin{titlepage}
\thispagestyle{empty}
\newcommand{\HRule}[2]{\centering\rule{#1}{#2}}
\newlength{\logowidth}
\setlength{\logowidth}{2.5cm}
\newlength{\logohspace}
\setlength{\logohspace}{\linewidth}
\addtolength{\logohspace}{-3\logowidth}

\begin{center}
	
%	\large{2013-2014}

			
	%%%%%%%%%%%% Titre
	\begin{minipage}{0.8\linewidth}
	\vspace{3cm}
		\HRule{\linewidth}{0.5mm}\\
		\vspace{0.5em}
		\sc\Large
		Marcel Dynamique
		\HRule{\linewidth}{0.5mm}
	\end{minipage}
	%%%%%%%%%%%%
	\vspace{8cm}
%%	\vfill

	\large{\sc{Business Plan}}
		
	
%	\vfill
% 	\includegraphics[height=5cm]{table.png} 
\end{center}

\end{titlepage}    


%Chapitre I : Présentation du projet et de l'équipe de management
%Chapitre II : Présentation de l'offre
%Chapitre III : Présentation du marché et de la concurrence
%Chapitre IV : Présentation du projet de développement
%Chapitre V : Analyse des risques
%Chapitre VI : Hypothèses financières et prévisions
%Chapitre VII : Le document de synthèse (Executive summary)


\section{Présentation du projet et de l'équipe de management}
L'objectif du projet porté par la société Marcel est de démocratiser la robotique de laboratoire.
L'énoncé de cet objectif seul soulève plusieurs questions : Qu'entend-on par laboratoire ? Quel place y trouve-t-on pour le vaste domaine de la robotique ?
Que voulons-nous dire par démocratiser, pourquoi, et comment comptons-nous y parvenir ?\\

\subsection{Présentation du projet}

Le terme laboratoire fait généralement référence au laboratoire de recherche scientifique.
Il s'agit d'un lieu où des équipes de chercheurs réalisent des expérimentations, souvent à l'aide de plateformes de recherche,
pour faire avancer la connaissance commune, ou y développent de nouvelles technologies.
Les plateformes de recherche peuvent prendre différents aspects, allant d'une suite logicielle comme Matlab à l'accélérateur de particules, en passant par le robot mobile programmable.
Il existe des laboratoires publiques, comme le CNRS, l'INRIA ou le CEA, mais également de nombreux laboratoire privés au sein d'entreprises comme Thalès, SAGEM, etc.

Cependant la dénomination de laboratoire peut être élargie. 
Par exemple, dans les universités des pays germaniques et anglo-saxons notamment, on appelle \emph{laboratory} les séances de Travaux Pratiques (TP).
Durant ces séances, les étudiants expérimentent, également par le biais de plateformes.
Ces expérimentations permettent aux étudiants de mieux appréhender et d'approfondir les notions abordées en cours magistraux.
Les plateformes utilisées ici peuvent prendre plusieurs formes.
Il peut s'agir simplement d'un ordinateur sur lequel un étudiant pourra écrire un programme, 
ou d'une carte électronique sur laquelle il devra souder des composants, ou encore des plateformes plus complexes, comme une chaine d'assemblage par exemple.

Depuis la fin des années 1990, des laboratoires plus populaires ont commencé à émerger,
il s'agit des Fab Labs (contraction de l'anglais \emph{fabrication laboratory}, pour \emph{laboratoire de fabrication}).
Ces lieux accueillent tout un chacun et disposent de l'outillage nécessaire pour réaliser des projets, les partager et d'expérimenter avec les nouvelles technologies.

Dans la suite de ce document, le terme laboratoire fera référence à son sens le plus général : un lieu prévu pour expérimenter.\\

La robotique peut s'inviter dans les laboratoires sous différents aspects.
On peut penser d'abord à des robots servant à assister l'expérimentateur, comme des bras manipulateurs en médecine ou des centrifugeuses en biologie par exemple, 
ou a des robots de fabrication, comme des systèmes de placement de composants électroniques ou encore des imprimantes 3D.
Ce n'est pas cet aspect de la robotique qui nous intéresse dans le cadre de notre projet.

Certains laboratoires se spécialisent dans l'étude, la conception ou encore la fabrication de robots.
On pense naturellement à Boston Dynamics, ou les robots présentés au Darpa Challenge. 
De nombreux clubs ou associations, dans un cadre scolaire ou non, se proposent également de concevoir et d'expérimenter autour de la robotique, 
en participant à des concours par exemple.

Enfin, le troisième exemple de robotique de laboratoire est constitué par les robots \textbf{utilisés} en laboratoire, 
\textbf{en tant que} plateforme d'expérimentation, de support à la recherche.
L'usage de tels robots peut être très varié, allant de la recherche en intelligence artificielle, 
en siences cognitives ou encore en intéractions hommes-machines, pour les laboratoires de recherche, 
à des robots plus modestes pour découvrir l'électronique, l'algorithmique ou approfondir ses connaissances en traitement du signal.

C'est précisément cet aspect de la robotique de laboratoire que nous souhaitons développer au sein de la société Marcel.\\

La démocratisation d'un produit, d'un service ou même d'un concept peut se faire selon différentes voies. 
Les voies les plus évidentes sont son accessibilité en terme de prix, sa diffusion, et sa publicité.
Il peut alors facilement se propager au sein de la population.

Cependant, ces aspects de la démocratisation sont un peu réducteurs.
Nous pensons que l'utilisateur d'un produit doit être en mesure de démonter ce dernier et de le décortiquer pour en comprendre les rouages.
Il peut alors librement le modifier, ou dans une moindre mesure le réparer.
En d'autres termes, l'utilisateur doit pouvoir s'approprier son produit.

C'est ce que nous entendons par \og{}démocratiser la robotique de laboratoire\fg{}.

\subsection{L'équipe}
Le duo fondateur de la société Marcel est constitué de Laurent Fiack et de Jérôme Fellus, respectivements docteur et doctorant de l'Université de Cergy-Pontoise.
Nous avons, le long de nos parcours scolaire, professionnel et scientifique acquis de nombreuses compétences techniques, dans des domaines scientifiques variés. 
Nous pensons ces compétences nécessaires et suffisantes pour venir à bout de toutes les problématiques que constituent la robotique de laboratoire.

En plus de ces compétences techniques et scientifiques, nous avons participé activement à plusieurs associations de plus ou moins grande envergure.
Au cours de cette vie associative, nous avons été amenés à gérer divers projets, et à diriger des équipes.
Nous avons également dû gérer des budgets et nous intéresser à de nombreux aspects légaux.\\

Au cours de nos cursus respectifs, nous avons souvent été confrontés à l'usage ou à la maintenance de plateformes pédagogiques et scientifiques.
Ce point nous parraît fondamental, dans le sens où il constitue notre force, mais aussi notre principale motivation dans ce projet.

De part nos statuts successifs d'étudiant et d'enseignant, nous avons observé une opportunité vis à vis de la nature de ces plateformes.
Nous pensons que l'utilisation d'un robot comme support pédagogique apportera de la valeur aux enseignements pratiques, de part sa nature pluri-disciplinaire.
Dans ce contexte, nous ne voyons pas la robotique comme une discipline à part entière, mais plutôt comme un regroupement d'une multitude de métiers.
La plateforme robotique doit alors couvrir toutes les thématiques des Sciences et Technologies de l'Information et des Communications (STIC),
et s'oppose ainsi aux plateformes plus spécialisées, utilisées habituellement en enseignement.\\

Dans un second temps, notre formation au métier de chercheur nous a permis de constater une forme d'inadéquation de l'offre vis à vis des besoins.
La recherche scientifique fait en effet face à un manque cruel de ressources humaines en ingénieurs et en techniciens titulaires dédiés aux plateformes de recherche.

Le financement de la recherche scientifique est organisé par projets, d'une durée typique de trois ans, alors qu'une thématique de recherche a vocation à s'inscrire dans un temps plus long.
Cette organisation impose le recours à des contrats à durée déterminée, courant sur la durée des projets, et pose des problèmes de continuité.\\

Fort de ces constats, nous identifions les besoins précis suivants :
\begin{itemize}
	\item Des plateformes abordables, ouvertes et modulaire,
	\item Une expertise technique exploitable à la demande à toute phase du développement d'une plateforme.
\end{itemize}

\section{Présentation de l'offre}
Pour répondre aux besoins mis en évidence au chapitre précédent, nous découpons notre offre selon deux axes :
une plateforme robotique modulaire et ouverte d'une part et un service de prestation de soutien aux plateformes à destination des laboratoires publiques d'autre part.
\subsection{Le robot marcel}
Le robot marcel est composée d'une base roulante autonome, à la quelle on peut ajouter divers périphériques, capteurs ou actionneurs.
%Explication et motivation du 100% Hackable et du "réparable plutôt qu'incassable"
%Ces choix impactent la conception, qui se veut itérative
%La pédagogie est également au centre de la conception (eg. pointe de test sur carte électronique)

Dans sa version actuelle, la base roulante dispose de deux roues motrices indépendantes et d'une bille omni-directionnelle stabilisatrice.
Elle mesure 15 centimètres de haut pour 30 centimètres de diamètre environ.
Elle se construit à base de bois de type MDF découpé au laser, technique courament utilisé dans les Fab Labs.
Ce choix est crucial car nous voulons mettre en avant une philosophie de \og{}Réparable plutôt qu'incassable\fg{}.

%Mécanique
%- deux roues motrices indépendantes et une bille stabilisatrice
%- environ 15cm de haut et 30cm de diamètre
%- Capteurs d'évitement d'obstacle


%\begin{wrapfigure}{R}{0.4\textwidth}
%	\centering
%	\includegraphics[width=\linewidth]{\figpath/robot.png}
%	\caption{\label{fig:frog1}Base roulante Marcel : Prototye \og{}Mireille\fg{}.}
%\end{wrapfigure}

%Circuit reconfigurable = haute performance/W + grande valeur pédagogique

%La plateforme Marcel
% - Architecture numérique hautes performances (CPU+FPGA)
% - Électronique reconfigurable
% - Caméra haute résolution (haute vitesse/hyperspectrale/thermique)
% - Interopérable (ROS, YARP,...)
% - Modulaire (connectique standard)
% - Composants standards
% - Réparable plutôt qu'incassable
% - Abordable
% - 100% hackable
%Communauté Marcel
% - Schématique ouverte
% - Tutoriels en ligne gratuits
% - Supports de TP collaboratifs
% - Approvisionnement direct-fournisseur
\subsection{Les prestations}

\section{Présentation du marché et de la concurrence}
\section{Présentation du projet de développement}
\section{Analyse des risques}
\section{Hypothèses financières et prévisions}
\section{Le document de synthèse (Executive summary)}

%\section{Executive Summary}
%(Résumé) : Une page ou deux grand max
%\section{Les créateurs}
%Bilan personnel, points forts points faibles
%\section{Le projet}
%\section{Le Business Model}
%\subsection{Étude de marché}
%Étude documentaire (panorama du marché), ouverture internationale
%\subsection{Étude de la concurrence}
%Concurrents directs, concurrents indirects
%\subsection{Premier positionnement}
%Couple produit-marché
%\subsection{L'étude terrain}
%\subsection{Chiffre d'affaires prévisionnel}
%\subsection{Le Business Model}
%\section{La technologie et sa protection}
%\section{La stratégie marketing}
%\section{Les moyens à mettre en oeuvre}
%\section{Le statut juridique}
%\section{La structure du capital/Budget prévisionnel}
%\section{Le chiffrage du projet}
%\section{Les perspectives d'avenir}

\end{document}
