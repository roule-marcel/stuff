\documentclass[a4paper,12pt]{report}
\usepackage[utf8x]{inputenc}
\usepackage{palatino}
\usepackage{titlesec} %Nécessaire pour supprimer le mot chapitre
\usepackage[T1]{fontenc}
%\usepackage[francais]{babel}
\usepackage[french]{babel}
%\usepackage[english]{babel}
\usepackage[autolanguage,np]{numprint}
\usepackage{graphicx} %Pour insérer des images
\usepackage{float}
\usepackage{amsmath} 
\usepackage[absolute]{textpos}
\usepackage{textcomp}
\usepackage{multirow}
\usepackage[top=3.5cm,bottom=3.5cm,left=2.5cm,right=2.5cm]{geometry}
\usepackage{hyperref}
\usepackage{indentfirst}
\titleformat{\chapter}[hang]{\bf\huge}{\thechapter}{2pc}{} %Supprime le mot chapitre
\usepackage{eurosym}
\usepackage{wrapfig}

\usepackage{color}

\renewcommand{\arraystretch}{1.5}
\setlength{\tabcolsep}{0.5cm}

\makeatletter\renewcommand{\thesection}{\@arabic\c@section}
\makeatother

\newcommand{\figpath}{figures}

\newcommand{\address}{Neuvitec 95, Mail Gay Lussac -- 95000 Neuville-sur-Oise, FRANCE}

\begin{document}

\begin{titlepage}
\thispagestyle{empty}
\newcommand{\HRule}[2]{\centering\rule{#1}{#2}}
\newlength{\logowidth}
\setlength{\logowidth}{2.5cm}
\newlength{\logohspace}
\setlength{\logohspace}{\linewidth}
\addtolength{\logohspace}{-3\logowidth}

\begin{center}
	
%	\large{2013-2014}

			
	%%%%%%%%%%%% Titre
	\begin{minipage}{0.8\linewidth}
	\vspace{3cm}
		\HRule{\linewidth}{0.5mm}\\
		\vspace{0.5em}
		\sc\Large
		Marcel Dynamique
		\HRule{\linewidth}{0.5mm}
	\end{minipage}
	%%%%%%%%%%%%
	\vspace{8cm}
%%	\vfill

	\large{\sc{Business Plan}}
		
	
%	\vfill
% 	\includegraphics[height=5cm]{table.png} 
\end{center}

\end{titlepage}    


%Chapitre I : Présentation du projet et de l'équipe de management
%Chapitre II : Présentation de l'offre
%Chapitre III : Présentation du marché et de la concurrence
%Chapitre IV : Présentation du projet de développement
%Chapitre V : Analyse des risques
%Chapitre VI : Hypothèses financières et prévisions
%Chapitre VII : Le document de synthèse (Executive summary)


\section{Présentation du projet et de l'équipe de management}
%Ce premier chapitre est fondamental. A coté de la fiche d’identité de l’entreprise et d’une brève description de son métier, 
%les investisseurs veulent connaître l’équipe de direction et savoir s’ils peuvent leur faire confiance pour mener à bien le projet... 
%-Quelles sont les compétences réunies ? 
%-Sont elles suffisantes pour couvrir tous les besoins de l’entreprise ? 
%-Quelle est l’expérience de l’équipe dans ce secteur ou sur ces produits ou services ? 
%-A-t-on identifié les compétences clés ne font pas partie de l’équipe initiale ? 
%-L’équipe a-t-elle déjà travaillé ensemble ? 
%-Le rôle de chacun est il défini (organisation) ? Y a-t-il un leader ? 
%-Quel sera leur apport en capital ? 
%-Quelle est la motivation de chacun ?
%Enfin, dès ce premier chapitre, il convient de préciser en quelques lignes l’ambition voulue par l'équipe pour l'entreprise. 
%Ce n’est pas une stratégie mais un objectif à atteindre. 

L'objectif du projet porté par la société Marcel est de démocratiser la robotique de laboratoire.
L'énoncé de cet objectif seul soulève plusieurs questions : Qu'entend-on par laboratoire ? Quel place y trouve-t-on pour le vaste domaine de la robotique ?
Que voulons-nous dire par démocratiser, pourquoi, et comment comptons-nous y parvenir ?\\

\subsection{Présentation du projet}

Le terme laboratoire fait généralement référence au laboratoire de recherche scientifique.
Il s'agit d'un lieu où des équipes de chercheurs réalisent des expérimentations, souvent à l'aide de plateformes de recherche,
pour faire avancer la connaissance commune, ou y développent de nouvelles technologies.
Les plateformes de recherche peuvent prendre différents aspects, allant d'une suite logicielle comme Matlab à l'accélérateur de particules, en passant par le robot mobile programmable.
Il existe des laboratoires publiques, comme le CNRS, l'INRIA ou le CEA, mais également de nombreux laboratoire privés au sein d'entreprises comme Thalès, SAGEM, etc.

Cependant la dénomination de laboratoire peut être élargie. 
Par exemple, dans les universités des pays germaniques et anglo-saxons notamment, on appelle \emph{laboratory} les séances de Travaux Pratiques (TP).
Durant ces séances, les étudiants expérimentent, également par le biais de plateformes.
Ces expérimentations permettent aux étudiants de mieux appréhender et d'approfondir les notions abordées en cours magistraux.
Les plateformes utilisées ici peuvent prendre plusieurs formes.
Il peut s'agir simplement d'un ordinateur sur lequel un étudiant pourra écrire un programme, 
ou d'une carte électronique sur laquelle il devra souder des composants, ou encore des plateformes plus complexes, comme une chaine d'assemblage par exemple.

Depuis la fin des années 1990, des laboratoires plus populaires ont commencé à émerger,
il s'agit des Fab Labs (contraction de l'anglais \emph{fabrication laboratory}, pour \emph{laboratoire de fabrication}).
Ces lieux accueillent tout un chacun et disposent de l'outillage nécessaire pour réaliser des projets, les partager et d'expérimenter avec les nouvelles technologies.

Dans la suite de ce document, le terme laboratoire fera référence à son sens le plus général : un lieu prévu pour expérimenter.\\

La robotique peut s'inviter dans les laboratoires sous différents aspects.
On peut penser d'abord à des robots servant à assister l'expérimentateur, comme des bras manipulateurs en médecine ou des centrifugeuses en biologie par exemple, 
ou a des robots de fabrication, comme des systèmes de placement de composants électroniques ou encore des imprimantes 3D.
Ce n'est pas cet aspect de la robotique qui nous intéresse dans le cadre de notre projet.

Certains laboratoires se spécialisent dans l'étude, la conception ou encore la fabrication de robots.
On pense naturellement à Boston Dynamics, ou les robots présentés au Darpa Challenge. 
De nombreux clubs ou associations, dans un cadre scolaire ou non, se proposent également de concevoir et d'expérimenter autour de la robotique, 
en participant à des concours par exemple.

Enfin, le troisième exemple de robotique de laboratoire est constitué par les robots \textbf{utilisés} en laboratoire, 
\textbf{en tant que} plateforme d'expérimentation, de support à la recherche.
L'usage de tels robots peut être très varié, allant de la recherche en intelligence artificielle, 
en siences cognitives ou encore en intéractions hommes-machines, pour les laboratoires de recherche, 
à des robots plus modestes pour découvrir l'électronique, l'algorithmique ou approfondir ses connaissances en traitement du signal.

C'est précisément cet aspect de la robotique de laboratoire que nous souhaitons développer au sein de la société Marcel.\\

La démocratisation d'un produit, d'un service ou même d'un concept peut se faire selon différentes voies. 
Les voies les plus évidentes sont son accessibilité en terme de prix, sa diffusion, et sa publicité.
Il peut alors facilement se propager au sein de la population.

Cependant, ces aspects de la démocratisation sont un peu réducteurs.
Nous pensons que l'utilisateur d'un produit doit être en mesure de démonter ce dernier et de le décortiquer pour en comprendre les rouages.
Il peut alors librement le modifier, ou dans une moindre mesure le réparer.
En d'autres termes, l'utilisateur doit pouvoir s'approprier son produit.

C'est ce que nous entendons par \og{}démocratiser la robotique de laboratoire\fg{}.

\subsection{L'équipe}
Le duo fondateur de la société Marcel est constitué de Laurent Fiack et de Jérôme Fellus, respectivements docteur et doctorant de l'Université de Cergy-Pontoise.
Nous avons, le long de nos parcours scolaire, professionnel et scientifique acquis de nombreuses compétences techniques, dans des domaines scientifiques variés. 
Nous pensons ces compétences nécessaires et suffisantes pour venir à bout de toutes les problématiques que constituent la robotique de laboratoire.

En plus de ces compétences techniques et scientifiques, nous avons participé activement à plusieurs associations de plus ou moins grande envergure.
Au cours de cette vie associative, nous avons été amenés à gérer divers projets, et à diriger des équipes.
Nous avons également dû gérer des budgets et nous intéresser à de nombreux aspects légaux.\\

Au cours de nos cursus respectifs, nous avons souvent été confrontés à l'usage ou à la maintenance de plateformes pédagogiques et scientifiques.
Ce point nous parraît fondamental, dans le sens où il constitue notre force, mais aussi notre principale motivation dans ce projet.

De part nos statuts successifs d'étudiant et d'enseignant, nous avons observé une opportunité vis à vis de la nature de ces plateformes.
Nous pensons que l'utilisation d'un robot comme support pédagogique apportera de la valeur aux enseignements pratiques, de part sa nature pluri-disciplinaire.
Dans ce contexte, nous ne voyons pas la robotique comme une discipline à part entière, mais plutôt comme un regroupement d'une multitude de métiers.
La plateforme robotique doit alors couvrir toutes les thématiques des Sciences et Technologies de l'Information et des Communications (STIC),
et s'oppose ainsi aux plateformes plus spécialisées, utilisées habituellement en enseignement.\\

Dans un second temps, notre formation au métier de chercheur nous a permis de constater une forme d'inadéquation de l'offre vis à vis des besoins.
La recherche scientifique fait en effet face à un manque cruel de ressources humaines en ingénieurs et en techniciens titulaires dédiés aux plateformes de recherche.

Le financement de la recherche scientifique est organisé par projets, d'une durée typique de trois ans, alors qu'une thématique de recherche a vocation à s'inscrire dans un temps plus long.
Cette organisation impose le recours à des contrats à durée déterminée, courant sur la durée des projets, et pose des problèmes de continuité.\\

Fort de ces constats, nous identifions les besoins précis suivants :
\begin{itemize}
	\item Des plateformes abordables, ouvertes et modulaire,
	\item Une expertise technique exploitable à la demande à toute phase du développement d'une plateforme.
\end{itemize}

\section{Présentation de l'offre}
%Avant de présenter le marché, qui fera l’objet du chapitre 3, il convient de détailler ce que l’on va y vendre et de quelle façon. 
%%Décrire le ou les métiers de l'entreprise ; 
%%Définir la valeur pour le client :
%Il faut préciser quels sont les segments clients visés, comment ils sont définis et quelles sont leurs attentes. 
%Il faut démontrer comment le produit ou service répond mieux que ce qui existe (avantages concurrentiels) aux attentes clients. Quelle valeur est apportée au client.
%Il faut expliquer ce qui sera fait pour fidéliser les clients dans le long terme et à quel coût.
%%Présenter le produit ou le service :
%Il faut décrire l’état d’avancement des éventuels développements.
%Il faut présenter les partenariats clés nécessaires ou utiles pour le développement ou la commercialisation. A quel stade en sont ils ?
%Il faut décrire les obligations de maintenance, de service après vente et de garantie à mettre en œuvre. 
%Il faut prévoir les phases suivantes du produit et les ressources nécessaires pour chaque phase. 
%%Décrire la situation en terme de protection intellectuelle et de respect de la réglementation :
%Les avantages concurrentiels sont ils clairs et durablement défendables par un brevet par exemple. 
%Est-ce que toutes les caractéristiques du produit sont conformes à la réglementation
%%Définir la politique commerciale et marketing : 
%Il faut couvrir les 4P (Produits, Prix, Place (distribution), Promotion) en détail et convaincre que la stratégie de pénétration du marché permettra d’atteindre les parts de marché anticipées.

Pour répondre aux besoins mis en évidence au chapitre précédent, nous découpons notre offre selon deux axes :
une plateforme robotique modulaire et ouverte d'une part et un service de prestation de soutien aux plateformes à destination des laboratoires publiques d'autre part.
\subsection{Le robot Marcel}
%TODO points manquants:
%%Segments client
%%Positionnement face à la concurrence
%%Fidélisation client
%%Partenariats
%%Évolution (-> conception agile, yo)
%%Brevet gnagnagna, on va nous casser les couilles la dessus
Une question récurrente lorsque l'on parle de robotique concerne les usages : \og{}Il fait quoi votre robot ?\fg{}.
Question à laquelle nous répondons simplement : \og{}Rien !\fg{}.

%Citation (approximative) de JCC et al. : "Expliquez à quoi ça sert, ce qu'on peut faire avec, pas comment c'est fait, à quoi ça ressemble". 
%Ici, le "comment c'est fait" (caractéristiques, ) fait partie intégrante de "ce qu'on peut faire avec" (usage).
L'usage du robot Marcel découle directement de sa nature de plateforme d'expérimentation.
Il n'est pas pré-programmé, et ne dispose que d'un nombre limité de capteurs et d'actionneurs.
En revanche, il est livré avec toute la documentation nécessaire pour permettre à un étudiant ou à un chercheur de le programmer et de le faire évoluer à l'aide de différents modules.
Pour permettre au mieux sa démocratisation, il est programmable avec n'importe quel langage de programmation, depuis tous les types de machines.
%Ça c'est vraiment un point cool par rapport au reste
%TODO Décrire que c'est un "ordinateur avec des roues" ?

Le robot devient alors une véritable éprouvette, il appartient donc à son utilisateur de se l'approprier et d'en imaginer les missions, les usages.\\

Il nous parraît évident que nos robots seront manipulés de manière intensive, et qu'ils seront largement malmenés, l'erreur faisant partie intégrante de l'apprentissage tout comme de la recherche.
Nous pensons qu'il est illusoire d'espérer créer un produit indestructible face à cet usage, aussi nous préférons mettre en avant une philosophie de \og{}Réparable plutôt qu'incassable\fg{}.

Dans l'optique de la démocratisation expliquée au chapitre précédent, le robot doit être \emph{100\% Hackable}.
En d'autres termes, les utilisateurs doivent pouvoir facilement le démonter, en comprendre le fonctionnement, pour y apporter d'éventuelles modifications.
Ces choix imposent des contraintes supplémentaires, dès la conception du produit.
Nous essayons, dans cette optique, d'avoir au maximum recours à des composants standards, pour faciliter l'approvisionnement.
L'aspect pédagogique de la plateforme tient également une place centrale dans sa conception.
On trouvera par exemple des pointes de tests permettant de visualiser les signaux importants sur les cartes électroniques.\\

\begin{wrapfigure}{R}{0.4\textwidth}
	\centering
	\includegraphics[width=\linewidth]{\figpath/robot.png}
	\caption{\label{fig:marcel_mireille}Base roulante Marcel : Prototype \og{}Mireille\fg{}.}
\end{wrapfigure}

Le robot Marcel est composé d'une base roulante autonome, à laquelle on peut ajouter divers périphériques, capteurs ou actionneurs.

Dans sa version actuelle, la base roulante dispose de deux roues motrices indépendantes et d'une bille omni-directionnelle stabilisatrice.
Elle dispose de capteurs permettant d'éviter les obstacles.
Elle mesure 15 centimètres de haut pour 30 centimètres de diamètre environ.
Elle se construit à base de bois de type MDF découpé au laser, technique courament employée, et popularisée par les Fab Labs.

Le prototype \og{}Mireille\fg{}, adapté pour distribuer des friandises à l'occasion de la conférence IEEE ICDL-EPIROB, est représenté à la \figurename~\ref{fig:marcel_mireille}.\\

L'électronique de la base roulante est architecturée autour d'un circuit logique reconfigurable FPGA (\emph{Field Programmable Gate Array}).
Ce type de circuits est connu pour offrir de hautes performances, sous une consommation énergétique réduite, tout en garantissant flexibilité, extensibilité et modularité.
Il sont une bonne solution pour munir le robot d'une caméra, et de réaliser tous les traitements visuels dans le robot, tout en garantissant une certaine autonomie à ce dernier.
De plus, permettre à l'utilisateur la possibilité de personnaliser son architecture numérique, laissera plus de liberté quant à la personnalisation du robot en lui-même.\\

Mais le choix de ce type de composants est également guidé par la grande valeur pédagogique qu'ils véhiculent.
Les circuits FPGA permettent en effet de descendre beaucoup plus profondément dans la compréhension du fonctionnement des circuits numériques.
Ils permettent de mettre en évidence toutes les problématiques liées à la discipline de l'architecture des ordinateurs modernes.

%TODO logiciel, interopérabilité

%TODO communauté
% - Schématique ouverte
% - Tutoriels en ligne gratuits
% - Supports de TP collaboratifs
% - Approvisionnement direct-fournisseur
\subsection{Les prestations}

\section{Présentation du marché et de la concurrence}
%La connaissance du marché et de la concurrence est fondamentale pour réussir.
%Il faut convaincre l’investisseur que l’on connaît ce marché et qu’il est possible d’en capter une part significative. 
%%Présenter le marché : 
%Quelle est sa taille ? La croissance prévue ? Quels sont les facteurs qui l’influencent ? Quels sont les facteurs clés de succès sur ce marché ? Quels sont les facteurs décisifs d’achat ? Quelles sont les barrières à l’entrée ?
%%Segmenter le marché : 
%Comment se fait la répartition en volume ? Y a-t-il des attentes différentes des clients ? La concurrence est elle la même d’un segment à l’autre ? 
%%Etudier la concurrence :
%Qui sont ils ? Combien ? Quelle est leur part de marché et leur santé financière ? Quelles sont leurs forces et faiblesses ? Comment répondent ils aux facteurs clés de succès sur ce marché ? Quel est le positionnement de l’entreprise par rapport à la concurrence ? 
\section{Présentation du projet de développement}
%La construction d’un projet de développement passe nécessairement par plusieurs étapes à formaliser : 
%%L’analyse stratégique :
%A partir de l’étude de l’existant et de l’environnement, il s’agit de formaliser une stratégie, de présenter le «business model» et de préciser la mission, la vision et les valeurs de l’entreprise.
%%L’organisation de l’entreprise : 
%Comment sont et seront organisées les grandes fonctions de l’entreprise ? (construire un organigramme) Que choisit-on de sous traiter ? Qui seront les fournisseurs clés ? Les processus principaux doivent être décrits... 
%%Le planning de développement :
%Il convient de détailler les étapes importantes en précisant les délais (qui doivent être réalistes), les points de blocage possibles, le planning des recrutements et le planning des investissements. 
\section{Analyse des risques}
%Les risques potentiels doivent être listés de manière relativement large en établissant pour chacun leur impact potentiel sur le projet (délai, coût, résultat) et en présentant des contre-mesures possibles et surtout crédibles.
\section{Hypothèses financières et prévisions}
%Contrairement aux idées reçues, ce n’est pas forcément la partie la plus importante. 
%Elle reste néanmoins indispensable, mais ne doit être réalisée qu’à la fin de l’écriture du «BP». 
%Tous les chiffres présentés doivent être établis sur la base d’hypothèses réalistes et vérifiables qu’il est judicieux de détailler dans un tableau à part en les commentant. 
%Plusieurs scénarios peuvent être présentés et chiffrés (idéalement, le pire et le meilleur...) en tenant compte des éventuels retards... 
%Il faut raisonner en milliers d’euros et éviter les chiffres trop précis (une prévision de 300 K€ est plus crédible que 307 K€ !!!). 
%Quatre documents prévisionnels doivent être présentés : 
%%Un plan de trésorerie : 
%Il doit être établi de façon mensuelle sur une période d’un an (ou deux si cela se justifie). 
%Il peut être complété d’un calcul du point mort ; 
%%Un compte de résultats :
%Il est présenté de façon annuelle, sur 3 à 5 exercices, et doit montrer comment le chiffre d’affaires couvre les charges, en grandes masses ; 
%%Un tableau de financement :
%Il est présenté de façon annuelle, sur 3 à 5 exercices et doit mettre en évidence l’évolution du fonds de roulement «FR» et du besoin en fonds de roulement «BFR» ;
%%Un bilan : 
%« Idem » sur 3 à 5 exercices... 
%Par ailleurs, il convient de préciser clairement le montant de la demande de financement et d’en expliquer les raisons et l’utilisation. 
%Il est recommandé de calculer le retour sur investissement envisagé pour l’investisseur en utilisant la méthode des cash flow actualisés ou « DCF » (voir dossier Acting N°6). 
\section{Le document de synthèse (Executive summary)}
%Encore appelé «Executive summary», il s’agit de présenter en 2 à 4 pages les points clés du projet. 
%Si la lecture de ces pages n’éveille pas l’intérêt, il est peu probable que l’investisseur potentiel aille plus loin. 
%Par contre, s’il est intéressé, il doit pouvoir aller facilement à n’importe quelle partie du «BP» pour fouiller plus avant un point qui l’intéresse, sans être obligé de tout lire à la suite. 
%Ce document est à placer au début du «BP», avant le sommaire, mais doit être rédigé à la fin, lorsque l’on maîtrise parfaitement son sujet. 

\end{document}
